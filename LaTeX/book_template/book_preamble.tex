\documentclass[12pt, a4paper, oneside]{ctexbook}
\usepackage{amsmath, amsthm, amssymb, bm, graphicx, mathrsfs, tikz, color, framed, bookmark, caption, geometry, enumitem}
\usepackage{hyperref} % 该宏包可能与其他宏包冲突,故放在所有引用的宏包之后

\usetikzlibrary{arrows.meta,decorations,calligraphy}

% 设置页边距
\geometry{a4paper, left=2cm, right=2cm, top=3cm, bottom=3cm}

\title{\Huge{\textbf{标题}}} % 标题
\author{作者} % 作者
\date{\today} % 日期
\linespread{1.5}

% 设置章节标题的格式
\ctexset{
    chapter = {
        name = {,},
        number = \arabic{chapter}
    }
}

% 文章所用图片在当前目录下的 Figures 目录
\graphicspath{{Figures/}}

% 对目录生成链接
\hypersetup{
    colorlinks = true, % 链接文字带颜色
    linkcolor = black, % 链接文字为黑色
    bookmarksopen = true, % 展开书签
    bookmarksnumbered = true, % 书签带章节编号
}

% 列表环境的格式
\setenumerate[1]{itemsep=0pt, partopsep=0pt, parsep=\parskip, topsep=0pt, left=2em}
\setitemize[1]{itemsep=0pt, partopsep=0pt, parsep=\parskip, topsep=0pt, left=2em}
\setdescription{itemsep=0pt, partopsep=0pt, parsep=\parskip, topsep=0pt}

% tikz 图形样式定义
\tikzset{
    % 空心圆点
    dot/.style = {
        draw,
        fill = white,
        circle,
        inner sep = 0pt,
        minimum size = 4pt,
    }
}

% 参考文献引用格式
\bibliographystyle{plain}

% 上标引用
\newcommand{\upcite}[1]{\textsuperscript{\cite{#1}}}

\captionsetup[table]{labelsep=quad} % 表格标题的分隔符
\captionsetup[figure]{labelsep=quad} % 图片标题的分隔符

% shaded 环境的背景颜色
\definecolor{shadecolor}{RGB}{231, 231, 231}

% 定理环境
% \newtheorem{theor}{定理}[chapter]
\newcounter{theoremname}[chapter] % 定理计数器
\renewcommand{\thetheoremname}{\arabic{chapter}.\arabic{theoremname}}
\newenvironment{theorem}[1][]{
    \begin{framed}
        \stepcounter{theoremname}
        \par\noindent
        \textbf{定理\arabic{chapter}.\arabic{theoremname}}#1\quad}
        % \begin{theor}[#1]}
{\end{framed}\par}

% 推论环境
\newcounter{corollaryname}[chapter] % 推论计数器
\newenvironment{corollary}[1][]{
    \begin{framed}
        \stepcounter{corollaryname}
        \par\noindent
        \textbf{推论\arabic{chapter}.\arabic{corollaryname}}#1\quad}
{\end{framed}\par}

% 引理环境
\newcounter{lemmaname}[chapter] % 引理计数器
\newenvironment{lemma}[1][]{
    \begin{framed}
        \stepcounter{lemmaname}
        \par\noindent
        \textbf{引理\arabic{chapter}.\arabic{lemmaname}}#1\quad}
{\end{framed}\par}

% 性质环境
\newcounter{propertyname} % 性质计数器
\newenvironment{property}[1][]{
    \begin{framed}
        \stepcounter{propertyname}
        \par\noindent
        \textbf{性质\arabic{propertyname}}#1\quad}
{\end{framed}\par}

% 定义环境
\newcounter{definitionname}[chapter] % 定义计数器
\newenvironment{definition}[1][]{
    \begin{framed}
        \stepcounter{definitionname}
        \par\noindent
        \textbf{定义\arabic{chapter}.\arabic{definitionname}}#1\quad}
{\end{framed}\par}

% 例题环境
\newcounter{problemname}[chapter] % 例题计数器
\newenvironment{problem}{
    \stepcounter{problemname}
    \par\noindent
    【\textbf{例\arabic{chapter}.\arabic{problemname}}】}
{\par}

% 题解环境
\newenvironment{solution}{\begin{shaded}\par\noindent\textbf{解:}}{\end{shaded}\par}

% 证明环境
\newenvironment{myproof}{\begin{shaded}\par\noindent\textbf{证明:}}{\qed\end{shaded}\par}

% 注记环境
\newenvironment{note}{\begin{shaded}\par\noindent\textbf{注:}}{\end{shaded}\par}