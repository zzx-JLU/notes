\documentclass[12pt, a4paper, oneside]{ctexbook}
\usepackage{amsmath, amsthm, amssymb, bm, graphicx, mathrsfs, tikz, color, framed, bookmark, caption, geometry}
\usepackage{hyperref} % 该宏包可能与其他宏包冲突,故放在所有引用的宏包之后

\usetikzlibrary{arrows.meta,decorations,calligraphy}

% 设置页边距
\geometry{a4paper, left=2cm, right=2cm, top=3cm, bottom=3cm}

\title{\Huge{\textbf{标题}}} % 标题
\author{作者} % 作者
\date{\today} % 日期
\linespread{1.5}

% 设置章节标题的格式
\ctexset{
    chapter = {
        name = {,},
        number = \arabic{chapter}
    }
}

% 文章所用图片在当前目录下的 Figures 目录
\graphicspath{{Figures/}}

% 对目录生成链接
\hypersetup{
    colorlinks = true, % 链接文字带颜色
    linkcolor = black, % 链接文字为黑色
    bookmarksopen = true, % 展开书签
    bookmarksnumbered = true, % 书签带章节编号
}

% tikz 图形样式定义
\tikzset{
    % 空心圆点
    dot/.style = {
        draw,
        fill = white,
        circle,
        inner sep = 0pt,
        minimum size = 4pt,
    }
}

% 参考文献引用格式
\bibliographystyle{plain}

% 上标引用
\newcommand{\upcite}[1]{\textsuperscript{\cite{#1}}}

\captionsetup[table]{labelsep=quad} % 表格标题的分隔符
\captionsetup[figure]{labelsep=quad} % 图片标题的分隔符

% shaded 环境的背景颜色
\definecolor{shadecolor}{RGB}{231, 231, 231}

% 定理环境
\newcounter{theoremname}[chapter] % 定理计数器
\newenvironment{theorem}[1][]{
    \begin{framed}
        \stepcounter{theoremname}
        \par\noindent
        \textbf{定理\arabic{chapter}.\arabic{theoremname}}#1\quad}
{\end{framed}\par}

% 推论环境
\newcounter{corollaryname}[chapter] % 推论计数器
\newenvironment{corollary}[1][]{
    \begin{framed}
        \stepcounter{corollaryname}
        \par\noindent
        \textbf{推论\arabic{chapter}.\arabic{corollaryname}}#1\quad}
{\end{framed}\par}

% 引理环境
\newcounter{lemmaname}[chapter] % 引理计数器
\newenvironment{lemma}[1][]{
    \begin{framed}
        \stepcounter{lemmaname}
        \par\noindent
        \textbf{引理\arabic{chapter}.\arabic{lemmaname}}#1\quad}
{\end{framed}\par}

% 性质环境
\newcounter{propertyname} % 性质计数器
\newenvironment{property}[1][]{
    \begin{framed}
        \stepcounter{propertyname}
        \par\noindent
        \textbf{性质\arabic{propertyname}}#1\quad}
{\end{framed}\par}

% 定义环境
\newcounter{definitionname}[chapter] % 定义计数器
\newenvironment{definition}[1][]{
    \begin{framed}
        \stepcounter{definitionname}
        \par\noindent
        \textbf{定义\arabic{chapter}.\arabic{definitionname}}#1\quad}
{\end{framed}\par}

% 例题环境
\newcounter{problemname}[chapter] % 例题计数器
\newenvironment{problem}{
    \stepcounter{problemname}
    \par\noindent
    【\textbf{例\arabic{chapter}.\arabic{problemname}}】}
{\par}

% 题解环境
\newenvironment{solution}{\begin{shaded}\par\noindent\textbf{解:}}{\end{shaded}\par}

% 证明环境
\newenvironment{myproof}{\begin{shaded}\par\noindent\textbf{证明:}}{\qed\end{shaded}\par}

% 注记环境
\newenvironment{note}{\begin{shaded}\par\noindent\textbf{注:}}{\end{shaded}\par}

% 文档开始
\begin{document}

    % 显示标题
    \maketitle

    % 前言部分的页码,小写罗马字母,从 1 开始
    \pagenumbering{roman}
    \setcounter{page}{1}
    
    % 前言标题
    \begin{center}
        \Huge\textbf{前言}
    \end{center}
    
    前言内容。再多一点,看一下首行缩进效果,红红火火恍恍惚惚或或或或或或或或或或或或或或或或或或或或或或或或或或或或。

    % 署名
    \begin{flushright}
        \begin{tabular}{c}
            姓名\\
            \today
        \end{tabular}
    \end{flushright}

    % 另起一页
    \newpage

    % 目录部分的页码,大写罗马字母,从 1 开始
    \pagenumbering{Roman}
    \setcounter{page}{1}

    % 显示目录
    \tableofcontents

    % 另起一页
    \newpage

    % 正文部分的页码,阿拉伯数字,从 1 开始
    \setcounter{page}{1}
    \pagenumbering{arabic}

    \chapter{章节标题}
    \section{二级标题}
    \subsection{三级标题}
    \subsubsection{四级标题}

    % 描述列表,可以指定描述词
    描述列表:
    \begin{description}
        \item[1.] (非负性)$d(x,y)\geqslant 0$且$d(x,y)=0\Leftrightarrow x=y$
        \item[2.] (对称性)$d(x,y)=d(y,x)$
        \item[3.] (三角不等式)$d(x,y)\leqslant d(x,z)+d(y,z)$
    \end{description}

    % 有序列表
    有序列表:
    \begin{enumerate}
        \item (非负性)$d(x,y)\geqslant 0$且$d(x,y)=0\Leftrightarrow x=y$
        \item (对称性)$d(x,y)=d(y,x)$
        \item (三角不等式)$d(x,y)\leqslant d(x,z)+d(y,z)$
    \end{enumerate}

    % 无序列表
    无序列表:
    \begin{itemize}
        \item $\textbf{n维欧氏空间} R^n$\\
        定义距离$d=(\sum_{k=1}^{n}\left | \xi _k-\eta _k\right
        |^2)^{1/2}$或者$d=\underset{1\leqslant k\leqslant n}{max}\left | \xi _k-\eta
        _k\right |$
        \item $\textbf{空间C[a,b]}$\\
        定义距离$d=\underset{a\leqslant t\leqslant  b }{max}\left | x(t)-y(t)\right |$
        \item $\textbf{空间L}^\infty$\\
        先回顾一下空间$L^\infty$:\begin{equation*}
            \left \| f\right \|_\infty=inf\begin{Bmatrix}
                M:\left | f\right |\leqslant M \quad a.e. \quad on\quad  E
            \end{Bmatrix}
        \end{equation*}
        \begin{equation*}
            L^\infty(E)=\begin{Bmatrix}
                f:f${在E可测}$ \left \| f\right \|_\infty< \infty		\end{Bmatrix}
        \end{equation*}
        定义距离$d=\underset{mF_0=0,F_0\subset F}{inf}\begin{Bmatrix}
            \underset{t\in F \setminus F_0}{sup}\left | x(t)-y(t)\right |
        \end{Bmatrix}$
    \end{itemize}
    
    % 定义环境示例
    \begin{definition}
        内容.
    \end{definition}

    % 注记环境示例
    \begin{note}
        注意了.
    \end{note}

    % 定理环境示例
    \begin{theorem}[($\textbf{唯一性}$)]
        $x_n \to x,x_n \to y\Rightarrow x=y$.
    \end{theorem}

    % 证明环境示例
    \begin{myproof}
        $0\leqslant d(x,y)\leqslant d(x_n,x)+d(x_n,y)\to 0$,根据夹逼定理,$d(x,y)=0 \Rightarrow x=y$
    \end{myproof}

    % 例题环境示例
    \begin{problem}
        微分方程解的存在性与唯一性:微分方程\begin{equation*}
            \left\{\begin{matrix}
                \frac{dy}{dx}=p(x,y)\\ 
                y(x_0)=y_0
            \end{matrix}\right.
        \end{equation*}
        其中$f\in C(\mathbb{R}^2)$\\
        设$y$满足Lipschitz条件,即$\exists K>0,s.t.$\begin{equation*}
            \left | f(x,y)-f(x,y')\right | \leqslant K\left | y-y'\right |
        \end{equation*}
    \end{problem}

    % 题解环境示例
    \begin{solution}
        \begin{align*}
            y(x)-y_0&=\int_{x_0}^{x}\frac{dy}{dx}dx\\
            &=\int_{x_0}^{x}f(x,y(x))dx\\
            &=\int_{x_0}^{x}f(t,y(t))dt
        \end{align*}
        (可以看出这个解的结构但无法说明解的存在性与唯一性,但是积分不一定收敛)

        取$\delta >0,s.t. k\delta <1$,在$C[x_0-\delta,x_0+\delta]$上定义T:\begin{equation*}
            (Ty)(x)=y_0+\int_{x_0}^{x}f(t,y(t))dt
        \end{equation*}
    \end{solution}

    % 性质环境示例
    \begin{property}[(有界性)]
        内容.
    \end{property}

    % 引理环境示例
    \begin{lemma}
        内容.
    \end{lemma}

    % 推论环境示例
    \begin{corollary}
        内容.
    \end{corollary}

    % 图片引入及引用
    如图 \ref{1} 所示.

    \begin{figure}[htbp]
        \centering % 居中
        \includegraphics[scale=0.2]{Ali.jpg} % 引入图片
        \caption{this is Ali} % 图片标题
        \label{1} % 标签
    \end{figure}

    % 表格定义及引用
    如表 \ref{table:1} 所示.

    \begin{table}[htbp]
        \caption{表格标题}
        \label{table:1} % 标签

        % 定义表格格式
        \centering % 居中

        % 表格内容
        \begin{tabular}{| l | c | c | r |} % | 表示垂直线,p 表示宽度,l 表示左对齐,c 表示居中对齐,r 表示右对齐
            \hline % 水平线
            \multicolumn{4}{|c|}{Country List}\\ % 多列合并
            \hline
            Country Name or Area Name & ISO ALPHA 2 Code & ISO ALPHA 3 & ISO ALPHA 4\\
            \hline
            Afghanistan & AF & AFG & abcd\\
            \hline
        \end{tabular}
    \end{table}

    % 函数图像示例
    \begin{tikzpicture}[thick]
        \def\U{*0.027*\textwidth}
        % 绘制坐标轴
        \draw[-{Stealth}](-2\U,0)--(18\U,0) coordinate[label={right:$x$}];
        \draw[-{Stealth}](0,-2\U)--(0,13\U) coordinate[label={above:$y$}];
        % 设定坐标点
        \path
          coordinate(x) at (11\U,0)
          coordinate(y) at (0,7\U)
          coordinate(c) at (11\U,7\U)
          coordinate(c1) at (7\U,5\U)
          coordinate(c2) at (15\U,9\U);
        \draw[dashed](y)--(c)--(x);
        % 绘制曲线
        \draw(7\U,5\U) cos(11\U,7\U) sin(15\U,9\U);
        % 绘制各点标签
        \draw
          (y)node[dot,label={left:$y$}]{}
          (c)node[dot]{}
          (x)node[dot,label={below:$x$}]{}
          (0,0)node[dot,label={below left:$0$}]{};
    \end{tikzpicture}

    % 几何图像示例
    \begin{tikzpicture}
        \draw (0,0) circle(2);
        \fill (0,0) circle (.1);
        \node at (0,0) [right] {$x_k$};
        \draw (0,0) circle(1);
        \fill (0.5,0.5) circle (.1);
        \node at (0.5,0.5) [right] {$x_k+1$};
        \draw[red] (0.5,0.5) circle(1);
        \fill[red](0.7,0.7) circle (.1);
        \node at (0.7,0.7) [right] {$x$};
    \end{tikzpicture}

    % 引用参考文献
    引文.\upcite{1}

    \begin{property}[(可列可加性)]
        内容.
    \end{property}

    \begin{theorem}
        内容.
    \end{theorem}

    \chapter{第二章}

    \begin{theorem}
        内容.
    \end{theorem}

    \setcounter{propertyname}{0}
    \begin{property}
        内容.
    \end{property}

    \begin{property}[(非负性)]
        内容.
    \end{property}

    % 参考文献
    \bibliography{books}
\end{document}
