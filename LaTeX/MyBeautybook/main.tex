% !TeX root = main.tex
\documentclass[zihao=-4,fontset=windows]{MyBeautybook-CN}

% 参考文献数据库
\addbibresource{ref.bib}

\begin{document}
    \title{\Huge{\textbf{概率论与数理统计}}} % 标题
    \author{赵子轩} % 作者
    \date{\today} % 日期
    \linespread{1.5}
    \maketitle
    \thispagestyle{empty}

    \frontmatter
    \pagenumbering{Roman}
    \tableofcontents
    \newpage
    \thispagestyle{empty}

    \mainmatter

    \partabstract{\hspace{2em}本书系统地论述了微分几何的基本知识. 作者用前3章,  以及第6章共计4章的篇幅介绍了流形、多重线性函数、向量场、外微分、李群和活动标架等基本知识和工具.  基于上述基础知识,  论述了微分几何的核心问题,  即联络、黎曼几何、以及曲面论. 第7章是当前十分活跃的研究领域——复流形. 陈省身先生是此研究领域的大家,  此章包含有作者独到、深刻的见解和简捷、有效的方法. 第8章的Finsler几何是本书第2版新增加的一章,  它是陈省身先生近年来一直倡导的研究课题,  其中Chern联络具有突出的性质,  它使得黎曼几何成为Finsler几何的特殊情形. 最后两个附录,  介绍了大范围曲线论和曲面论,  以及微分几何与理论物理关系的论述,  为这两个活跃的前沿领域提出了不少进一步的研究课题.

    \hspace*{2em} 本书的作者之一是已故数学家陈省身先生,  他开创并领导着整体微分几何、纤维丛微分几何、“陈省身示性类”等领域的研究,  他是第一个获得世界数学界最高荣誉“沃尔夫奖”的华人,  被称为“当今最伟大的数学家”,  被国际数学界尊为“微分几何之父”. }
    \part{微分几何讲义一陈省身}

    \chapter{随机事件及其概率}

    \section{随机试验}

    \subsection{频率}

    \subsubsection{四级标题}

    在一定条件下必然出现的现象叫做\textbf{必然现象}.在相同的条件下,可能出现不同的结果,而在试验或观测之前不能预知确切结果的现象叫做\textbf{随机现象}.

    \begin{enumerate}
        \item 可重复性:试验可以在相同条件下重复进行多次,甚至进行无限次;
        \item 可观测性:每次试验的所有可能结果都是明确的、可以观测的,并且试验的可能结果有两个或两个以上;
        \item 随机性:每次试验出现的结果是不确定的,在试验之前无法预先确定究竟会出现哪一个结果,
    \end{enumerate}

    \begin{definition}
        设 $\varOmega$ 为样本空间,$\mathcal{F}$ 为 $\varOmega$ 的某些子集组成的集合类.如果 $\mathcal{F}$ 满足:
        \begin{enumerate}
            \item $\varOmega \in \mathcal{F}$;
            \item 若 $A \in \mathcal{F}$,则 $\overline{A} \in \mathcal{F}$;
            \item 若 $A_i \in \mathcal{F}, \, i=1,2,\cdots$,则 $\displaystyle\bigcup_{i=1}^\infty A_i \in \mathcal{F}$,
        \end{enumerate}
        则称 $\mathcal{F}$ 为一个\textbf{事件域},也称为 $\sigma$ \textbf{域}或 $\sigma$ \textbf{代数}.将 $(\varOmega, \mathcal{F})$ 称为\textbf{可测空间}.
    \end{definition}

    $$
    y=2x+3
    $$

    \begin{theorem}
        若 $P$ 是 $\mathcal{F}$ 上满足 $P(\varOmega) = 1$ 的非负集合函数,则 $P$ 具有可列可加性的充分必要条件是:\\
        (1)$P$ 是有限可加的;(2)$P$ 是下连续的.
    \end{theorem}

    \begin{lemma}
        若 $P$ 是 $\mathcal{F}$ 上满足 $P(\varOmega) = 1$ 的非负集合函数,则 $P$ 具有可列可加性的充分必要条件是:\\
        (1)$P$ 是有限可加的;(2)$P$ 是下连续的.
    \end{lemma}

    \begin{proof}
        令abc123,则 $xy=12$.
    \end{proof}

    \begin{solution}
        内容.
    \end{solution}

    \begin{proposition}
        内容.
    \end{proposition}

    \begin{theorem}[][名称]
        内容.
    \end{theorem}

    \begin{example}
        内容.
    \end{example}

    \begin{corollary}
        内容.

        占位

        占位

        占位

        占位

        占位

        占位

        占位

        占位

        占位

        占位

        占位
    \end{corollary}

    \begin{example}
        内容.

        内容
    \end{example}

    \begin{proof}
        因为
        $$
        x=1,
        $$
        所以
        $$
        y=ax+b.
        $$

        啊啊啊啊啊啊啊啊啊啊啊啊啊啊啊啊啊啊啊啊啊啊啊啊啊啊啊啊啊啊啊啊啊啊啊啊啊啊啊啊啊啊啊啊啊啊啊啊啊啊啊啊啊啊啊啊啊啊啊啊
    \end{proof}

    \begin{conclusion}
        内容.
    \end{conclusion}

    \begin{conclusion}
        内容.
    \end{conclusion}

    \begin{property}
        \begin{equation}
            abc
        \end{equation}
    \end{property}

    \begin{property}
        \begin{equation}
            def
        \end{equation}
    \end{property}

    \begin{property}
        \begin{equation}
            123
        \end{equation}
    \end{property}

    \section{2}

    \begin{note}
        内容.

        \indent 长内容,观察换行和分段效果。哈哈哈哈哈哈哈哈哈哈哈哈哈哈哈哈哈哈哈哈哈哈哈哈哈哈哈哈哈哈哈哈哈哈哈哈哈哈哈哈哈哈哈哈哈 \supercite{Huybrechts2010Complex}
    \end{note}

    \begin{axiom}
        内容.

        $$
        \Diff[x]{y}
        $$

        $$
        \Dif{x}{y}
        $$

        $$
        x\pr \dd
        $$

        $$
        x'
        $$
    \end{axiom}

    \section{3}

    \let\cleardoublepage\clearpage

    \chapter{二}

    \chapter{嘿嘿abc123}

    \chapter{嘿嘿\texorpdfstring{$x=2$}{}}

    \backmatter
    \appendix % 附录章节

    \normalem
    \printbibliography[
        heading = bibintoc,
        title = {参考文献}
    ]
    \printindex
    \thispagestyle{empty}
\end{document}