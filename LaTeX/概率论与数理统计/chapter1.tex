\chapter{随机事件及其概率}

\section{随机试验}

如果试验具有以下特点:
\begin{enumerate}
    \item 可重复性:试验可以在相同条件下重复进行多次,甚至进行无限次;
    \item 可观测性:每次试验的所有可能结果都是明确的、可以观测的,并且试验的可能结果有两个或两个以上;
    \item 随机性:每次试验出现的结果是不确定的,在试验之前无法预先确定究竟会出现哪一个结果,
\end{enumerate}
则称之为\textbf{随机试验},简称为\textbf{试验}.

通常用字母 $E$ 表示一个随机试验. 随机试验 $E$ 的基本结果称为\textbf{样本点},用 $\omega$ 表示.称随机试验 $E$ 的所有基本结果的集合为\textbf{样本空间},用 $\varOmega = \{ \omega \}$ 表示.

\section{随机事件}

\subsection{随机事件的概念}

随机试验 $E$ 的样本空间 $\varOmega = \{ \omega \}$ 的子集称为随机试验 $E$ 的\textbf{随机事件},简称为\textbf{事件},用大写字母 $A,B,C$ 等表示.

设 $A \subseteq \varOmega$,如果试验结果 $\omega \in A$,则称在这次试验中事件 $A$ 发生;如果 $\omega \notin A$,则称事件 $A$ 不发生.

由一个样本点 $\omega$ 组成的事件称为\textbf{基本事件}.

样本空间 $\varOmega$ 本身也是 $\varOmega$ 的子集,它包含 $\varOmega$ 的所有样本点,在每次试验中 $\varOmega$ 必然发生,称为\textbf{必然事件}.

空集 $\text{\O}$ 也是 $\varOmega$ 的子集,它不包含任何样本点,在每次试验中都不可能发生,称为\textbf{不可能事件}.

在一个样本空间中,如果只有有限个样本点,则称它为\textbf{有限样本空间};如果有无限个样本点,则称它为\textbf{无限样本空间}.

\subsection{随机事件的关系}

\subsubsection{事件的包含}

如果当事件 $A$ 发生时事件 $B$ 一定发生,则称事件 $B$ \textbf{包含}事件 $A$,记作 $A \subseteq B$.

对于任意事件 $A$,有 $\text{\O} \subseteq A \subseteq \varOmega$.

如果 $A \subseteq B,B \subseteq C$,则 $A\subseteq C$.

\subsubsection{事件的相等}

如果事件 $A$ 和事件 $B$ 相互包含,即 $A \subseteq B$ 且 $B \subseteq A$,则称事件 $A$ 与事件 $B$ \textbf{相等},记作 $A=B$.

\subsubsection{事件的互不相容}

如果事件 $A$ 和事件 $B$ 在同一次试验中不能同时发生,则称事件 $A$ 与事件 $B$ 是\textbf{互不相容}的,或称事件 $A$ 与事件 $B$ 是\textbf{互斥}的.

\subsubsection{事件的互逆}

如果在每一次试验中事件 $A$ 和事件 $B$ 必有一个且仅有一个发生,则称事件 $A$ 与事件 $B$ 是\textbf{互逆}的或\textbf{对立}的,称其中的一个事件是另一个事件的\textbf{逆事件},记作 $\overline{A}=B$,或 $\overline{B}=A$.

显然,$\overline{\overline{A}}=A$.

\subsection{随机事件的运算}

\subsubsection{事件的并}

如果事件 $A$ 和事件 $B$ 至少有一个发生,则这样的一个事件称为事件 $A$ 与事件 $B$ 的\textbf{并事件}或\textbf{和事件},记作 $A \cup B$.
$$
A \cup B = \{ \omega \mid \omega \in A \;\text{或}\; \omega \in B \}
$$

事件 $A$ 和事件 $B$ 作为样本空间 $\varOmega$ 的子集,并事件 $A \cup B$ 就是子集 $A$ 与 $B$ 的并集.

对于任何事件 $A$ 与 $B$,有
\begin{gather*}
    A \cup A = A \\
    A \cup \text{\O} = A \\
    A \cup B = B \cup A \\
    A \cup \overline{A} = \varOmega \\
    A \subseteq A \cup B \\
    B \subseteq A \cup B
\end{gather*}

如果 $A \subseteq B$,则有 $A \cup B=B$.

事件的并可以推广到多个事件的情形:
\begin{gather*}
    \bigcup_{i=1}^n A_i = \{ \text{事件} A_1,A_2,\cdots,A_n \text{中至少有一个发生} \} \\
    \bigcup_{i=1}^\infty A_i = \{ \text{事件} A_1,A_2,\cdots,A_n,\cdots \text{中至少有一个发生} \}
\end{gather*}

\subsubsection{事件的交}

如果事件 $A$ 和事件 $B$ 同时发生,则这样的一个事件称为事件 $A$ 与事件 $B$ 的\textbf{交事件}或\textbf{积事件},记作 $A \cap B$ 或 $AB$.
$$
A \cap B = \{ \omega \mid \omega \in A \;\text{且}\; \omega \in B \}
$$

事件 $A$ 和事件 $B$ 作为样本空间 $\varOmega$ 的子集,交事件 $A \cap B$ 就是子集 $A$ 与 $B$ 的交集.

对于任何事件 $A$ 与 $B$,有
\begin{gather*}
    A \cap A = A\\
    A \cap \text{\O} = \text{\O}\\
    A \cap B = B \cap A\\
    A \cap \overline{A} = \text{\O}\\
    A \cap B \subseteq A\\
    A \cap B \subseteq B
\end{gather*}

如果 $A \subseteq B$,则有 $A \cap B=A$. 如果 $A$ 与 $B$ 互不相容,则有 $A \cap B = \text{\O}$.

事件的交可以推广到多个事件的情形:
\begin{gather*}
    \bigcap_{i=1}^n A_i = \{ \text{事件} A_1,A_2,\cdots,A_n \text{同时发生} \}\\
    \bigcap_{i=1}^\infty A_i = \{ \text{事件} A_1,A_2,\cdots,A_n,\cdots \text{同时发生} \}
\end{gather*}

\subsubsection{事件的差}

如果事件 $A$ 发生而事件 $B$ 不发生,则这样的一个事件称为事件 $A$ 与事件 $B$ 的\textbf{差事件},记作 $A-B$.
$$
A - B = \{ \omega \mid \omega \in A \;\text{且}\; \omega \notin B \}
$$

对于任何事件 $A$ 与 $B$,有
\begin{gather*}
    A - A = \text{\O}\\
    A - \text{\O} = A\\
    A - B = A - AB = A \overline{B}\\
    \varOmega - A = \overline{A}\\
    A - \varOmega = \text{\O}\\
    (A-B) \cup B = A \cup B
\end{gather*}