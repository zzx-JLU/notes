\chapter{随机事件及其概率}

\section{随机试验}

在一定条件下必然出现的现象叫做\textbf{必然现象}.在相同的条件下,可能出现不同的结果,而在试验或观测之前不能预知确切结果的现象叫做\textbf{随机现象}.

随机现象具有随机性和统计规律性.

\begin{itemize}
    \item 随机性:对随机现象进行观测时,不能预先确定其结果.
    \item 统计规律性:对随机现象进行大量重复观测后,其结果往往会表现出某种规律性.
\end{itemize}

为了研究和揭示随机现象的统计规律性,需要对随机现象进行大量重复的观察、测量或试验,统称为试验.

如果试验具有以下特点:
\begin{enumerate}
    \item 可重复性:试验可以在相同条件下重复进行多次,甚至进行无限次;
    \item 可观测性:每次试验的所有可能结果都是明确的、可以观测的,并且试验的可能结果有两个或两个以上;
    \item 随机性:每次试验出现的结果是不确定的,在试验之前无法预先确定究竟会出现哪一个结果,
\end{enumerate}
则称之为\textbf{随机试验},简称为\textbf{试验}.

通常用字母 $E$ 表示一个随机试验. 随机试验 $E$ 的基本结果称为\textbf{样本点},用 $\omega$ 表示.随机试验 $E$ 的所有基本结果的集合称为\textbf{样本空间},用 $\varOmega = \{ \omega \}$ 表示.

\section{随机事件}

\subsection{随机事件的概念}

随机试验 $E$ 的样本空间 $\varOmega = \{ \omega \}$ 的子集称为随机试验 $E$ 的\textbf{随机事件},简称为\textbf{事件},用大写字母 $A,B,C$ 等表示.

设 $A \subseteq \varOmega$,如果试验结果 $\omega \in A$,则称在这次试验中事件 $A$ 发生;如果 $\omega \notin A$,则称事件 $A$ 不发生.

由一个样本点 $\omega$ 组成的事件称为\textbf{基本事件}.

样本空间 $\varOmega$ 本身也是 $\varOmega$ 的子集,它包含 $\varOmega$ 的所有样本点,在每次试验中 $\varOmega$ 必然发生,称为\textbf{必然事件}.

空集 $\text{\O}$ 也是 $\varOmega$ 的子集,它不包含任何样本点,在每次试验中都不可能发生,称为\textbf{不可能事件}.

在一个样本空间中,如果只有有限个样本点,则称它为\textbf{有限样本空间};如果有无限个样本点,则称它为\textbf{无限样本空间}.

\subsection{随机事件的关系}

\subsubsection{事件的包含}

如果当事件 $A$ 发生时事件 $B$ 一定发生,则称事件 $B$ \textbf{包含}事件 $A$,记作 $A \subseteq B$.

对于任意事件 $A$,有 $\text{\O} \subseteq A \subseteq \varOmega$.

如果 $A \subseteq B,B \subseteq C$,则 $A\subseteq C$.

\subsubsection{事件的相等}

如果事件 $A$ 和事件 $B$ 相互包含,即 $A \subseteq B$ 且 $B \subseteq A$,则称事件 $A$ 与事件 $B$ \textbf{相等},记作 $A=B$.

\subsubsection{事件的互不相容}

如果事件 $A$ 和事件 $B$ 在同一次试验中不能同时发生,则称事件 $A$ 与事件 $B$ 是\textbf{互不相容}的,或称事件 $A$ 与事件 $B$ 是\textbf{互斥}的.

任意两个基本事件一定互斥.

\subsubsection{事件的互逆}

如果在每一次试验中事件 $A$ 和事件 $B$ 必有一个且仅有一个发生,则称事件 $A$ 与事件 $B$ 是\textbf{互逆}的或\textbf{对立}的,称其中的一个事件是另一个事件的\textbf{逆事件},记作 $\overline{A}=B$,或 $\overline{B}=A$.

显然,$\overline{\overline{A}}=A$.

\subsection{随机事件的运算}

\subsubsection{事件的并}

如果事件 $A$ 和事件 $B$ 至少有一个发生,则这样的一个事件称为事件 $A$ 与事件 $B$ 的\textbf{并事件}或\textbf{和事件},记作 $A \cup B$.
$$
A \cup B = \{ \omega \mid \omega \in A \;\text{或}\; \omega \in B \}
$$

事件 $A$ 和事件 $B$ 作为样本空间 $\varOmega$ 的子集,并事件 $A \cup B$ 就是子集 $A$ 与 $B$ 的并集.

对于任何事件 $A$ 与 $B$,有
\begin{gather*}
    A \cup A = A \\
    A \cup \text{\O} = A \\
    A \cup B = B \cup A \\
    A \cup \overline{A} = \varOmega \\
    A \subseteq A \cup B \\
    B \subseteq A \cup B
\end{gather*}

如果 $A \subseteq B$,则有 $A \cup B=B$.

事件的并可以推广到多个事件的情形:
\begin{gather*}
    \bigcup_{i=1}^n A_i = \{ \text{事件} A_1,A_2,\cdots,A_n \text{中至少有一个发生} \} \\
    \bigcup_{i=1}^\infty A_i = \{ \text{事件} A_1,A_2,\cdots,A_n,\cdots \text{中至少有一个发生} \}
\end{gather*}

\subsubsection{事件的交}

如果事件 $A$ 和事件 $B$ 同时发生,则这样的一个事件称为事件 $A$ 与事件 $B$ 的\textbf{交事件}或\textbf{积事件},记作 $A \cap B$ 或 $AB$.
$$
A \cap B = \{ \omega \mid \omega \in A \;\text{且}\; \omega \in B \}
$$

事件 $A$ 和事件 $B$ 作为样本空间 $\varOmega$ 的子集,交事件 $A \cap B$ 就是子集 $A$ 与 $B$ 的交集.

对于任何事件 $A$ 与 $B$,有
\begin{gather*}
    A \cap A = A\\
    A \cap \text{\O} = \text{\O}\\
    A \cap B = B \cap A\\
    A \cap \overline{A} = \text{\O}\\
    A \cap B \subseteq A\\
    A \cap B \subseteq B
\end{gather*}

如果 $A \subseteq B$,则有 $A \cap B=A$.

如果 $A$ 与 $B$ 互不相容,则有 $A \cap B = \text{\O}$.

事件的交可以推广到多个事件的情形:
\begin{gather*}
    \bigcap_{i=1}^n A_i = \{ \text{事件} A_1,A_2,\cdots,A_n \text{同时发生} \}\\
    \bigcap_{i=1}^\infty A_i = \{ \text{事件} A_1,A_2,\cdots,A_n,\cdots \text{同时发生} \}
\end{gather*}

\subsubsection{事件的差}

如果事件 $A$ 发生而事件 $B$ 不发生,则这样的一个事件称为事件 $A$ 与事件 $B$ 的\textbf{差事件},记作 $A-B$.
$$
A - B = \{ \omega \mid \omega \in A \;\text{且}\; \omega \notin B \}
$$

对于任何事件 $A$ 与 $B$,有
\begin{gather*}
    A - A = \text{\O}\\
    A - \text{\O} = A\\
    A - B = A - AB = A \overline{B}\\
    \varOmega - A = \overline{A}\\
    A - \varOmega = \text{\O}\\
    (A-B) \cup B = (B-A) \cup A = A \cup B\\
    A \cup B = A \cup (B-AB) = B \cup (A-AB)
\end{gather*}

$A-B,AB,B-A$ 两两互斥,且 $A \cup B = (A-B) \cup AB \cup (B-A)$,$A = (A-B) \cup AB$,$B = (B-A) \cup AB$.

\subsubsection{随机事件的运算规律}

\begin{enumerate}
    \item 交换律:$A \cup B = B \cup A$,$AB=BA$.
    \item 结合律:$(A \cup B) \cup C = A \cup (B \cup C)$,$(AB)C=A(BC)$.
    \item 分配律:$A(B \cup C)=(AB)\cup(AC)$,$A\cup(BC)=(A \cup B)(A \cup C)$.
    \item 对偶律:$\overline{A \cup B}=\overline{A}\,\overline{B}$,$\overline{AB}=\overline{A}\cup\overline{B}$.
\end{enumerate}

对于多个随机事件,以上的运算规律也成立.

\section{随机事件的概率}

\subsection{频率}

\begin{definition}
    设在相同的条件下进行的 $n$ 次试验中,事件 $A$ 发生了 $n_A$ 次,则称 $n_A$ 为事件 $A$ 发生的\textbf{频数},称比值 $\dfrac{n_A}{n}$ 为事件 $A$ 发生的\textbf{频率},记作 $f_n(A)$,即
    $$
    f_n(A)=\dfrac{n_A}{n}
    $$
\end{definition}

事件 $A$ 发生的频率反映了事件 $A$ 在 $n$ 次试验中发生的频繁程度.频率越大,表明事件 $A$ 的发生越频繁,从而可知事件 $A$ 在一次试验中发生的可能性越大.

频率的基本性质:

\setcounter{propertyname}{0}

\begin{property}[(非负性)]
    对于任意事件 $A$,有 $f_n(A) \geqslant 0$.
\end{property}

\begin{property}[(规范性)]
    对于必然事件 $\varOmega$,有 $f_n(\varOmega)=1$.
\end{property}

\begin{property}[(有限可加性)]
    对于两两互不相容的事件 $A_1,A_2,\cdots,A_m$(即当 $i\not=j$ 时,有 $A_i A_j = \text{\O}$,$i,j=1,2,\cdots,m$),有
    $$
    f_n(\bigcup_{i=1}^m A_i) = \sum_{i=1}^m f_n(A_i)
    $$
\end{property}

在相同的条件下重复进行 $n$ 次试验,当 $n$ 增大时,事件 $A$ 发生的频率 $f_n(A)$ 呈现出稳定性,逐渐稳定于某一常数 $p$.用这一常数表示事件 $A$ 发生的可能性大小,称为事件 $A$ 的概率,记为 $P(A)$,即 $P(A)=p$.

当 $n$ 很大时,可以用频率 $f_n(A)$ 作为概率 $P(A)$ 的近似值.

\subsection{概率}

\begin{definition}
    设随机试验 $E$ 的样本空间为 $\varOmega$,如果对于 $E$ 的每一个事件 $A$,有唯一的实数 $P(A)$ 和它对应,并且这个事件的函数 $P(A)$ 满足以下条件:
    \begin{enumerate}
        \item 非负性:对于任意事件 $A$,有 $P(A) \geqslant 0$;
        \item 规范性:对于必然事件 $\varOmega$,有 $P(\varOmega)=1$;
        \item 可列可加性:对于两两互不相容的事件 $A_1,A_2,\cdots$,有
        $$
        P(\bigcup_{i=1}^\infty A_i) = \sum_{i=1}^\infty P(A_i)
        $$
    \end{enumerate}
    则称 $P(A)$ 为事件 $A$ 的\textbf{概率}.
\end{definition}

\setcounter{propertyname}{0}

\begin{property} \label{property:1}
    对于不可能事件 $\text{\O}$,有 $P(\text{\O})=0$.
\end{property}

\begin{myproof}
    因为 $\text{\O} = \text{\O} \cup \text{\O} \cup \cdots$,根据概率的可列可加性,有
    $$
    P(\text{\O}) = P(\text{\O}) + P(\text{\O}) + \cdots
    $$

    由概率的非负性知 $P(\text{\O}) \geqslant 0$,因此 $P(\text{\O})=0$.
\end{myproof}

\begin{property}[(有限可加性)] \label{property:2}
    对于两两互不相容的事件 $A_1,A_2,\cdots,A_n$,有
    $$
    P(\bigcup_{i=1}^n A_i) = \sum_{i=1}^n P(A_i)
    $$
\end{property}

\begin{myproof}
    令 $A_i = \text{\O} \;(i=n+1,n+2,\cdots)$,根据概率的可列可加性及性质\ref{property:1},有
    $$
    P(\bigcup_{i=1}^n A_i) = P(\bigcup_{i=1}^\infty A_i) = \sum_{i=1}^\infty P(A_i) = \sum_{i=1}^n P(A_i)
    $$
\end{myproof}

\begin{property}
    对于任一事件 $A$,有 $P(\overline{A})=1-P(A)$.
\end{property}

\begin{myproof}
    因为 $A \cup \overline{A} = \varOmega$,且 $A \overline{A} = \text{\O}$,由性质 2 及概率的规范性,得
    $$
    P(\varOmega) = P(A \cup \overline{A}) = P(A) + P(\overline{A}) = 1
    $$
    即
    $$
    P(\overline{A})=1-P(A)
    $$
\end{myproof}

\begin{property}
    如果 $A \subseteq B$,则有 $P(B-A)=P(B)-P(A)$,$P(A) \leqslant P(B)$.
\end{property}

\begin{myproof}
    因为 $A \subseteq B$,从而有 $B = A \cup (B-A)$,且 $A(B-A)=\text{\O}$,由性质 2 可得
    $$
    P(B) = P(A \cup (B-A)) = P(A) + P(B-A)
    $$
    所以
    $$
    P(B-A)=P(B)-P(A)
    $$

    由于 $P(B-A) \geqslant 0$,因此
    $$
    P(A) \leqslant P(B)
    $$
\end{myproof}

\begin{property}
    对于任一事件 $A$,有 $P(A) \leqslant 1$.
\end{property}

\begin{myproof}
    因为 $A \subseteq \varOmega$,由性质 4 及概率的规范性,可得
    $$
    P(A) \leqslant P(\varOmega) = 1
    $$
\end{myproof}

\begin{property}[(概率的减法公式)]
    对于任意两个事件 $A$ 与 $B$,有 $P(B-A)=P(B)-P(AB)$.
\end{property}

\begin{myproof}
    由于 $B-A=B-AB$,而 $AB \subseteq B$,根据性质 4 可得
    $$
    P(B-A)=P(B-AB)=P(B)-P(AB)
    $$
\end{myproof}