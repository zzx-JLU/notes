% !TeX root = main.tex

\chapter{随机变量及其分布}
\thispagestyle{plain}

\section{随机变量及其分布函数}

\section{离散型随机变量及其概率分布}

\question 将3个球随机地放入4个杯子中,求杯子中球的最大个数 $X$ 的概率分布.

\begin{solution}
    将3个相同的球随机地放入4个不同的杯子中,共有 $4^3$ 种情况.
    
    $X$ 所有可能的取值为1,2,3.当 $X=1$ 时, 3个球被放入不同的杯子中,第一个球放入4个杯子中的任意一个,第二个球放入剩下3个杯子中的任意一个,第三个球放入剩下2个杯子中的任意一个,共有 $4 \times 3 \times 2 = 24$ 种情况.因此
    $$
    P(X=1) = \dfrac{24}{4^3} = \dfrac{3}{8}
    $$
    当 $X=3$ 时, 3个球被放入同一个杯子中,此时有4种情况,因此
    $$
    P(X=3) = \dfrac{4}{4^3} = \dfrac{1}{16}
    $$
    由于 $P(X=1) + P(X=2) + P(X=3) = 1$,所以
    $$
    P(X=2) = 1 - P(X=1) - P(X=3) = 1 - \dfrac{3}{8} - \dfrac{1}{16} = \dfrac{9}{16}
    $$
    综上, $X$ 的概率分布为
    \begin{center}
        \begin{tabular}{c | c c c}
            \hline
            $X$ & $1$ & $2$ & $3$ \\
            \hline
            $P$ \rule{0pt}{20pt} & $\dfrac{3}{8}$ & $\dfrac{9}{16}$ & $\dfrac{1}{16}$ \\[6pt]
            \hline
        \end{tabular}
    \end{center}
\end{solution}

\section{连续型随机变量及其概率密度函数}

\section{常用的概率分布}

\question[巴拿赫问题] 某人有两盒火柴,每盒都有 $n$ 根.每次使用时,任取一盒并从中抽出一根.当此人第一次抽到空盒时,另一盒中恰有 $r$ ($0 \leqslant r \leqslant n$)根火柴的概率是多少?

\begin{solution}
    设两盒火柴分别为A, B,由对称性知,只需计算事件 $A=$ ``抽到A盒为空,此时B盒中恰有 $r$ 根火柴''的概率,所求概率是此概率的2倍.

    事件 $A$ 的发生可以分成两个阶段:前 $2n-r$ 次中A盒抽到 $n$ 次、B盒抽到 $n-r$ 次,第 $2n-r+1$ 次抽到A盒.由于每次抽到A盒或B盒的概率都是 $\dfrac{1}{2}$,所以
    $$
    P(A) = \mathrm{C}_{2n-r}^n \left( \dfrac{1}{2} \right)^n \left( \dfrac{1}{2} \right)^{n-r} \cdot \dfrac{1}{2} = \dfrac{\mathrm{C}_{2n-r}^n}{2^{2n-r+1}}
    $$
    进而可得所求概率为
    $$
    p = 2P(A) = \dfrac{\mathrm{C}_{2n-r}^n}{2^{2n-r}}
    $$
\end{solution}

\begin{note}
    \indent 巴拿赫问题的本质是:当第 $n+1$ 次抽到某盒火柴时,求总的抽取次数为 $2n-r+1$ 的概率.设随机变量 $X$ 为第 $n+1$ 次抽到A盒时的总抽取次数,则 $X$ 服从帕斯卡分布 $Nb(n+1, \dfrac{1}{2})$,因此有
    $$
    P(A) = P(X = 2n-r+1) = \mathrm{C}_{2n-r+1-1}^{n+1-1} \left( \dfrac{1}{2} \right)^{n+1} \left( \dfrac{1}{2} \right)^{2n-r+1-(n+1)} = \dfrac{\mathrm{C}_{2n-r}^n}{2^{2n-r+1}}
    $$

    通过巴拿赫问题可以得到下列等式:
    $$
    \sum_{r=0}^n \dfrac{\mathrm{C}_{2n-r}^n}{2^{2n-r}} = 1
    $$
\end{note}