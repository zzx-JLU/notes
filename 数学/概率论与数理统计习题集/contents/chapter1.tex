% !TeX root = main.tex

\chapter{随机事件及其概率}
\thispagestyle{plain}

\section{随机事件}

\question 写出下列随机试验的样本空间:

(1)抛一枚硬币,观察正面和反面出现的情况;

(2)抛三枚硬币,观察正面和反面出现的情况;

(3)连续抛一枚硬币,直至出现正面为止;

(4)抛一枚骰子,观察出现的点数;

(5)抛两枚骰子,观察出现的点数;

(6)抛两枚骰子,记录出现的点数之和;

(7)在一个箱子中装有10个同型号的某种零件,其中有3个次品和7个合格品,从该箱子中任取3个零件,观察其中次品的个数;

(8)记录某机场在一天内收到咨询电话的次数;

(9)测试电视机的寿命;

(10)口袋中有黑、白、红球各一个,从中任取两个球;先从中取出一个,放回后再取出一个;

(11)口袋中有黑、白、红球各一个,从中任取两个球;先从中取出一个,不放回后再取出一个.

\begin{solution}
    (1) $\varOmega = \{ 0, 1 \}$,其中 $0$ 表示反面,$1$ 表示正面.

    (2) $\varOmega = \{ (0,0,0), (0,0,1), (0,1,0), (0,1,1), (1,0,0), (1,0,1), (1,1,0), (1,1,1) \}$

    (3) $\varOmega = \{ (1), (0,1), (0,0,1), (0,0,0,1), \cdots \}$

    (4) $\varOmega = \{ 1,2,3,4,5,6 \}$

    (5) $\varOmega = \{ (x,y) \mid x,y = 1,2,3,4,5,6 \}$

    (6) $\varOmega = \{ 2, 3, 4, \cdots, 12 \}$

    (7) $\varOmega = \{ 0,1,2,3 \}$

    (8) $\varOmega = \{ 0,1,2,3,\cdots \}$

    (9) $\varOmega = [0, +\infty)$

    (10) $\varOmega = \{ \text{黑黑}, \text{黑白}, \text{黑红}, \text{白黑}, \text{白白}, \text{白红}, \text{红黑}, \text{红白}, \text{红红} \}$

    (11) $\varOmega = \{ \text{黑白}, \text{黑红}, \text{白黑}, \text{白红}, \text{红黑}, \text{红白} \}$
\end{solution}

\question 设 $A,B,C$ 为三事件,试表示下列事件:

(1) $A$ 发生,$B,C$ 不发生;

(2) $A,B,C$ 都发生;

(3) $A,B,C$ 都不发生;

(4) $A,B,C$ 中只有一个发生;

(5) $A,B,C$ 中至少有一个发生;

(6) $A,B,C$ 中至多有一个发生;

(7) $A,B,C$ 中至少有一个不发生;

(8) $A,B,C$ 中至多有两个发生;

(9) $A,B,C$ 中至少有两个发生;

(10) $A,B,C$ 中恰好有两个发生.

\begin{solution}
    (1) $A \, \overline{B} \, \overline{C}$

    (2) $ABC$

    (3) $\overline{A} \, \overline{B} \, \overline{C}$

    (4) $A \, \overline{B} \, \overline{C} \cup \overline{A} B \overline{C} \cup \overline{A} \, \overline{B} \, C$

    (5) $\varOmega - \overline{A} \, \overline{B} \, \overline{C} = \overline{\overline{A} \, \overline{B} \, \overline{C}} = A \cup B \cup C$

    (6) $\overline{A} \, \overline{B} \, \overline{C} \cup A \, \overline{B} \, \overline{C} \cup \overline{A} B \overline{C} \cup \overline{A} \, \overline{B} \, C$

    (7) $\overline{A} \cup \overline{B} \cup \overline{C}$

    (8) $\varOmega - ABC = \overline{ABC} = \overline{A} \cup \overline{B} \cup \overline{C}$

    (9) $AB \cup AC \cup BC$

    (10) $AB \overline{C} \cup A \overline{B} C \cup \overline{A} BC$
\end{solution}

\question 判断下列命题是否成立:

(1) $A-(B-C) = (A-B) \cup C$;

(2)若 $AB = \text{\O}$ 且 $C \subseteq A$,则 $BC = \text{\O}$;

(3) $(A \cup B) - B = A$;

(4) $(A - B) \cup B = A$.

\begin{solution}
    (1)
    $$
    \begin{aligned}
        A-(B-C) &= A - B \overline{C} \\
        &= A \overline{B \overline{C}} \\
        &= A (\overline{B} \cup C) \\
        &= (A \overline{B}) \cup (AC) \\
        &= (A-B) \cup (AC) \\
        & \not= (A-B) \cup C
    \end{aligned}
    $$
    命题1不成立.

    (2)成立.

    (3)
    $$
    (A \cup B) - B = (A \cup B) \overline{B} = (A \overline{B}) \cup (B \overline{B}) = A \overline{B} \not= A
    $$
    命题3不成立.

    (4)
    $$
    (A - B) \cup B = (A \overline{B}) \cup B = (A \cup B)(\overline{B} \cup B) = A \cup B \not= A
    $$
    命题4不成立.
\end{solution}

\section{随机事件的概率}

\question 设 $A,B$ 是同一个试验中的两个事件,$P(A)=0.6$,$P(A-B)=0.2$,$P(A \cup B) = 0.9$.求 $P(\overline{AB}), P(B), P((\overline{A} \cup B)(A \cup \overline{B}))$.

\begin{solution}
    由于 $P(A-B) = P(A) - P(AB)$,所以
    $$
    P(AB) = P(A) - P(A-B) = 0.6 - 0.2 = 0.4
    $$
    进而可得
    $$
    P(\overline{AB}) = 1 - P(AB) = 1 - 0.4 = 0.6
    $$

    由于 $P(A \cup B) = P(A) + P(B) - P(AB)$,所以
    $$
    P(B) = P(A \cup B) - P(A) + P(AB) = 0.9 - 0.6 + 0.4 = 0.7
    $$

    由随机事件的运算性质可得
    $$
    \begin{aligned}
        (\overline{A} \cup B)(A \cup \overline{B}) &= [(\overline{A} \cup B) A] \cup [(\overline{A} \cup B) \overline{B}] \\
        &= (A \overline{A}) \cup (AB) \cup (\overline{A} \, \overline{B}) \cup (B \overline{B}) \\
        &= (AB) \cup (\overline{A \cup B})
    \end{aligned}
    $$
    又因为 $(AB)(\overline{A \cup B}) = (AB)(\overline{A} \, \overline{B}) = \text{\O}$,所以
    $$
    \begin{aligned}
        P((\overline{A} \cup B)(A \cup \overline{B})) &= P((AB) \cup (\overline{A \cup B})) \\
        &= P(AB) + P(\overline{A \cup B}) \\
        &= P(AB) + 1 - P(A \cup B) \\
        &= 0.4 + 1 - 0.9 \\
        &= 0.5
    \end{aligned}
    $$
\end{solution}

\question 设 $A$ 和 $B$ 是同一试验 $E$ 的两个随机事件,证明:$1 - P(\overline{A}) - P(\overline{B}) \leqslant P(AB) \leqslant P(A \cup B)$.

\begin{proof}
    因为 $AB \subseteq A \subseteq (A \cup B)$,所以
    $$
    P(AB) \leqslant P(A \cup B)
    $$

    因为
    $$
    1 - P(AB) = P(\overline{AB}) = P(\overline{A} \cup \overline{B}) \leqslant P(\overline{A}) + P(\overline{B})
    $$
    所以
    $$
    1 - P(\overline{A}) - P(\overline{B}) \leqslant P(AB)
    $$
\end{proof}

\question 抛两枚硬币,求出现一个正面一个反面的概率.

\begin{solution}
    此试验的样本空间为 $\varOmega = \{ (\text{正}, \text{正}), (\text{正}, \text{反}), (\text{反}, \text{正}), (\text{反}, \text{反}) \}$,样本点的个数为4,且每个样本点发生的可能性是相等的.事件“出现一个正面一个反面”含有的样本点个数为2,根据古典概型可得该事件发生的概率为 $\dfrac{1}{2}$.
\end{solution}

\begin{note}
    \indent 如果将样本空间写成 $\varOmega' = \{ (\text{正}, \text{正}), (\text{反}, \text{反}), (\text{一正一反}) \}$,这3个样本点不是等可能的,不满足古典概型的条件.
\end{note}