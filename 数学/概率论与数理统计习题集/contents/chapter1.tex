% !TeX root = main.tex

\chapter{随机事件及其概率}
\thispagestyle{plain}

\section{随机事件}

\question 写出下列随机试验的样本空间:

(1)抛一枚硬币,观察正面和反面出现的情况;

(2)抛一枚骰子,观察出现的点数;

(3)在一个箱子中装有10个同型号的某种零件,其中有3个次品和7个合格品,从该箱子中任取3个零件,观察其中次品的个数;

(4)记录某机场在一天内收到咨询电话的次数;

(5)测试电视机的寿命;

(6)抛三枚硬币,观察正面和反面出现的情况;

(7)连续抛一枚硬币,直至出现正面为止;

(8)口袋中有黑、白、红球各一个,从中任取两个球;先从中取出一个,放回后再取出一个;

(9)口袋中有黑、白、红球各一个,从中任取两个球;先从中取出一个,不放回后再取出一个.

\begin{solution}
    (1)$\varOmega = \{ \omega_1, \omega_2 \}$,其中 $\omega_1$ 表示正面朝上,$\omega_2$ 表示反面朝上.

    (2)$\varOmega = \{ 1,2,3,4,5,6 \}$

    (3)$\varOmega = \{ 0,1,2,3 \}$

    (4)$\varOmega = \{ 0,1,2,3,\cdots \}$

    (5)$\varOmega = [0, +\infty)$

    (6)$\varOmega = \{ (0,0,0), (0,0,1), (0,1,0), (0,1,1), (1,0,0), (1,0,1), (1,1,0), (1,1,1) \}$,其中 $0$ 表示反面,$1$ 表示正面.

    (7)$\varOmega = \{ (1), (0,1), (0,0,1), (0,0,0,1), \cdots \}$

    (8)$\varOmega = \{ \text{黑黑, 黑白, 黑红, 白黑, 白白, 白红, 红黑, 红白, 红红} \}$

    (9)$\varOmega = \{ \text{黑白, 黑红, 白黑, 白红, 红黑, 红白} \}$
\end{solution}

\question 设 $A,B,C$ 为三事件,试表示下列事件:

(1)$A$ 发生,$B,C$ 不发生;

(2)$A,B,C$ 都发生;

(3)$A,B,C$ 都不发生;

(4)$A,B,C$ 中只有一个发生;

(5)$A,B,C$ 中至少有一个发生;

(6)$A,B,C$ 中至多有一个发生;

(7)$A,B,C$ 中至少有一个不发生;

(8)$A,B,C$ 中至多有两个发生;

(9)$A,B,C$ 中至少有两个发生;

(10)$A,B,C$ 中恰好有两个发生.

\begin{solution}
    (1)$A \, \overline{B} \, \overline{C}$

    (2)$ABC$

    (3)$\overline{A} \, \overline{B} \, \overline{C}$

    (4)$A \, \overline{B} \, \overline{C} \cup \overline{A} B \overline{C} \cup \overline{A} \, \overline{B} \, C$

    (5)$\varOmega - \overline{A} \, \overline{B} \, \overline{C} = \overline{\overline{A} \, \overline{B} \, \overline{C}} = A \cup B \cup C$

    (6)$\overline{A} \, \overline{B} \, \overline{C} \cup A \, \overline{B} \, \overline{C} \cup \overline{A} B \overline{C} \cup \overline{A} \, \overline{B} \, C$

    (7)$\overline{A} \cup \overline{B} \cup \overline{C}$

    (8)$\varOmega - ABC = \overline{ABC} = \overline{A} \cup \overline{B} \cup \overline{C}$

    (9)$AB \cup AC \cup BC$

    (10)$AB \overline{C} \cup A \overline{B} C \cup \overline{A} BC$
\end{solution}

% \question 设 $A$ 和 $B$ 是同一试验 $E$ 的两个随机事件,证明:$1 - P(\overline{A}) - P(\overline{B}) \leqslant P(AB) \leqslant P(A \cup B)$.

% \begin{proof}
%     因为 $AB \subseteq A \subseteq (A \cup B)$,所以
%     $$
%     P(AB) \leqslant P(A \cup B)
%     $$
%     由概率的性质 \ref{prop:probability:add}、性质 \ref{prop:probability:converse} 及事件的对偶律,可得
%     $$
%     P(\overline{A}) + P(\overline{B}) \geqslant P(\overline{A} \cup \overline{B}) = P(\overline{AB}) = 1 - P(AB)
%     $$
%     因此
%     $$
%     1 - P(\overline{A}) - P(\overline{B}) \leqslant P(AB)
%     $$
% \end{proof}