% !TeX root = main.tex

\chapter{样本及样本函数的分布}

\section{总体与样本}

\subsection{总体}

在数理统计中,所研究对象的全体称为\textbf{总体},总体中的每个元素称为\textbf{个体}.

总体中所包含的个体总数叫做\textbf{总体容量}.如果一个总体的容量是有限的,则叫做\textbf{有限总体},否则叫做\textbf{无限总体}.

在具体问题中,人们关心的往往不是总体的一切方面,而是它的某一项数量指标 $X$ 以及它在总体中的分布情况.如果 $X$ 在总体中的分布情况可以用一个概率分布来表示,那么 $X$ 就可以看成是服从这一概率分布的随机变量.将表示总体的这项数量指标的随机变量 $X$ 可能取的值的全体作为总体,称作总体 $X$,而 $X$ 可能取的每一个数值 $x$ 都是一个个体.

总体 $X$ 的分布函数 $F(x)$ 叫做\textbf{总体的分布函数},总体 $X$ 的数字特征叫做\textbf{总体的数字特征}.

如果 $X$ 是离散型随机变量,则其概率分布叫做\textbf{离散型总体} $X$ \textbf{的概率分布};如果 $X$ 是连续型随机变量,则其概率密度叫做\textbf{连续型总体} $X$ \textbf{的概率密度};并将它们和总体 $X$ 的分布函数统称为\textbf{总体的分布}.

\subsection{简单随机样本}

从总体中抽取若干个个体的过程称为\textbf{抽样},抽样结果得到总体 $X$ 的一组试验数据(或观测值)称为\textbf{样本},样本中所含个体的数量称为\textbf{样本容量}.

为了使得样本能很好地反映总体的情况,抽样必须满足以下两个条件:
\begin{enumerate}
    \item 随机性:为了使样本具有充分的代表性,抽样必须是随机的,总体的每一个个体都有同等的机会被抽取到.
    \item 独立性:各次抽取必须是相互独立的,每次抽样的结果既不影响其他各次抽样的结果,也不受其他各次抽样结果的影响.
\end{enumerate}

满足随机性和独立性的抽样方法称为\textbf{简单随机抽样},由此得到的样本称为\textbf{简单随机样本}.

\begin{itemize}
    \item 从总体中进行放回抽样是简单随机抽样.
    \item 从无限总体中抽取一个个体后不会影响总体的分布,因此从无限总体中进行不放回抽样是简单随机抽样.
    \item 从有限总体中进行不放回抽样,当总体容量 $N$ 很大而样本容量 $n$ 较小时,可以近似看做放回抽样,即可以近似看做简单随机抽样.
\end{itemize}

从总体中抽取容量为 $n$ 的样本,就是对表示总体的随机变量 $X$ 随机地、独立地进行 $n$ 次试验,第 $i$ 次试验的结果可以看做一个随机变量 $X_i \, (i=1,2,\cdots,n)$,$n$ 次试验的结果就是 $n$ 个随机变量 $X_1,X_2,\cdots,X_n$,这些随机变量相互独立,并且与总体 $X$ 服从相同的分布.

\begin{definition}
    设总体 $X$ 是具有某一概率分布的随机变量,如果随机变量 $X_1,X_2,\cdots,X_n$ 相互独立,且都与 $X$ 具有相同的概率分布,则称 $X_1,X_2,\cdots,X_n$ 为来自总体 $X$ 的\textbf{简单随机样本},简称为\textbf{样本},$n$ 称为\textbf{样本容量}.在对总体 $X$ 进行一次具体的抽样并做观测后,得到样本 $X_1,X_2,\cdots,X_n$ 的确切数值 $x_1,x_2,\cdots,x_n$,称为\textbf{样本观察值}(或\textbf{观测值}),简称为\textbf{样本值}.
\end{definition}

如果从总体 $X$ 中抽取到样本观测值 $x_1,x_2,\cdots,x_n$,则可以认为是 $n$ 个相互独立的事件 $\{X_1=x_1\}, \{X_2=x_2\}, \cdots, \{X_n=x_n\}$ 同时发生了.

如果把样本容量为 $n$ 的样本看成是 $n$ 维随机变量 $(X_1,X_2,\cdots,X_n)$,则 $(X_1,X_2,\cdots,X_n)$ 所有可能取值的全体是 $n$ 维空间或它的一个子集,样本观测值是其中的一个点 $(x_1,x_2,\cdots,x_n)$.

如果总体 $X$ 的分布函数为 $F_X(t)$,则样本 $X_1,X_2,\cdots,X_n$ 的联合分布函数为
$$
F(t_1,t_2,\cdots,t_n) = F_X(t_1) \, F_X(t_2) \cdots F_X(t_n) = \prod_{i=1}^{n} F_X(t_i)
$$

如果总体 $X$ 是离散型随机变量,其概率分布为 $P \{ X=t \} = p_X(t)$,则样本 $X_1,X_2,\cdots,X_n$ 的联合概率分布为
$$
P \{ X_1=t_1, X_2=t_2, \cdots, X_n=t_n \} = p_X(t_1) \, p_X(t_2) \cdots p_X(t_n) = \prod_{i=1}^n p_X(t_i)
$$

如果总体 $X$ 是连续型随机变量,其概率密度为 $f_X(t)$,则样本 $X_1,X_2,\cdots,X_n$ 的联合概率密度为
$$
f(t_1,t_2,\cdots,t_n) = f_X(t_1) \, f_X(t_2) \cdots f_X(t_n) = \prod_{i=1}^n f_X(t_i)
$$

\section{直方图}

根据总体 $X$ 的样本观测值 $x_1,x_2,\cdots,x_n$ 作直方图的一般步骤为:

\begin{enumerate}
    \item 找出 $x_1,x_2,\cdots,x_n$ 中的最小值 $x_{(1)}$ 和最大值 $x_{(n)}$,选取略小于 $x_{(1)}$ 的数 $a$ 和略大于 $x_{(n)}$ 的数 $b$;

    \item 根据样本容量确定组数 $k$;

    \item 选取分点
    $$
    a = t_0 < t_1 < \cdots < t_{i-1} < t_i < \cdots < t_k = b
    $$
    把区间 $(a,b)$ 分为 $k$ 个子区间
    $$
    (a, t_1], (t_1, t_2], \cdots, (t_{i-1}, t_i], \cdots, (t_{k-1}, b]
    $$
    第 $i$ 个子区间 $(t_{i-1}, t_i]$ 的长度为
    $$
    \Delta t_i = t_i - t_{i-1}, \, i=1,2,\cdots,k
    $$
    各子区间的长度可以相等,也可以不相等.如果取各子区间的长度相等,则有
    $$
    \Delta t_i = \dfrac{b-a}{k}, \, i=1,2,\cdots,k
    $$
    记 $\Delta t = \dfrac{b-a}{k}$,并把 $\Delta t$ 叫做\textbf{组距}.此时分点为
    $$
    t_i = a + i \Delta t, \, i=1,2,\cdots,k
    $$
    为了方便起见,分点 $t_i$ 应比样本观测值 $x_i$ 多取一位有效数字;

    \item 数出 $x_1,x_2,\cdots,x_n$ 落在每个子区间 $(t_{i-1}, t_i]$ 内的频数 $n_i$,再算出频率
    $$
    f_i = \dfrac{n_i}{n}, \, i=1,2,\cdots,k
    $$

    \item 在 $Ox$ 轴上画出各个分点 $t_i(i=0,1,2,\cdots,k)$,并以各子区间 $(t_{i-1}, t_i]$ 为底,以 $y_i = \dfrac{f_i}{\Delta t_i} (i=0,1,2,\cdots,k)$ 为高作小矩形,这样作出的所有小矩形就构成了直方图.
\end{enumerate}

\vspace{0.5em}

第 $i$ 个矩形的面积为
$$
\Delta S_i = \Delta t_i \cdot \dfrac{f_i}{\Delta t_i} = f_i
$$
即第 $i$ 个小矩形的面积等于样本观测值落在该子区间内的频率,因此有
$$
\sum_{i=1}^k \Delta S_i = \sum_{i=1}^k f_i = 1
$$
即所有小矩形的面积的和等于 1.

当样本容量 $n$ 充分大时,随机变量 $X$ 落在第 $i$ 个小区间 $(t_{i-1}, t_i]$ 内的频率近似等于其概率,所以直方图大致反映了总体 $X$ 的概率分布.

\section{样本分布函数}

设总体 $X$ 的分布函数为 $F(x)$,从总体中抽取容量为 $n$ 的样本,样本观测值为 $x_1,x_2,\cdots,x_n$,其中相同的观测值可能重复出现若干次.假设在 $n$ 个样本观测值 $x_1,x_2,\cdots,x_n$ 中有 $k$ 个不相同的值,按由小到大的顺序依次记为 $x_{(1)} < x_{(2)} < \cdots < x_{(k)} \; (k \leqslant n)$,并假设各个 $x_{(i)}$ 出现的频数为 $n_i$,则各个 $x_{(i)}$ 出现的频率为
$$
f_i = \dfrac{n_i}{n}, \, i=1,2,\cdots,k, \, k \leqslant n
$$
显然有
$$
\sum_{i=1}^k n_i = n, \quad \sum_{i=1}^k f_i = 1
$$

设函数
$$
F_n(x) = \begin{cases}
    0 & x < x_{(1)} \\
    \displaystyle\sum_{j=1}^i f_j & x_{(i)} \leqslant x < x_{(i+1)} (i=1,2,\cdots,k-1) \\
    1 & x \geqslant x_{(k)}
\end{cases}
$$
把 $F_n(x)$ 叫做\textbf{样本分布函数}. $F_n(x)$ 的图像如图 \ref{fig:样本分布函数} 所示.

\begin{figure}[htbp]
    \centering

    \begin{tikzpicture}[>=Stealth, scale=4.2]
        % 坐标轴
        \draw[->] (-0.4, 0)--(1.2, 0) node[below]{$x$};
        \draw[->] (0, -0.2)--(0, 1.2) node[right]{$F_n(x)$};
        \node at (0, 0) [above left] {$O$};
        % 辅助线
        \draw[dashed] (-0.3, 0.125) -- (-0.3, 0) node[below]{$\scriptstyle{x_{(1)}}$};
        \draw[dashed] (-0.15, 0.25) -- (-0.15, 0) node[below]{$\scriptstyle{x_{(2)}}$};
        \draw[dashed] (0.1, 0.375) -- (0.1, 0) node[below]{$\scriptstyle{x_{(i)}}$};
        \draw[dashed] (0.3, 0.5) -- (0.3, 0) node[below]{$\scriptstyle{x_{(i+1)}}$};
        \draw[dashed] (0.45, 0.625) -- (0.45, 0);
        \draw[dashed] (0.6, 0.75) -- (0.6, 0);
        \draw[dashed] (0.75, 0.875) -- (0.75, 0);
        \draw[dashed] (0.9, 1) -- (0.9, 0) node[below]{$\scriptstyle{x_{(k)}}$};
        % 曲线
        \draw (-0.3, 0.125) -- (-0.15, 0.125) node[dot]{};
        \draw (-0.15, 0.25) -- (0.1, 0.25) node[dot]{};
        \draw (0.1, 0.375) -- (0.3, 0.375) node[dot]{};
        \draw (0.3, 0.5) -- (0.45, 0.5) node[dot]{};
        \draw (0.45, 0.625) -- (0.6, 0.625) node[dot]{};
        \draw (0.6, 0.75) -- (0.75, 0.75) node[dot]{};
        \draw (0.75, 0.875) -- (0.9, 0.875) node[dot]{};
        \draw (0.9, 1) -- (1.05, 1);
        % 刻度
        \draw (0, 1) node[left]{1} -- (0.05, 1);
    \end{tikzpicture}

    \caption{样本分布函数}
    \label{fig:样本分布函数}
\end{figure}

样本分布函数 $F_n(x)$ 的性质:

\begin{enumerate}
    \item $0 \leqslant F_n(x) \leqslant 1$.
    \item $F_n(x)$ 是单调不减函数.
    \item $F_n(-\infty) = 0$,$F_n(+\infty) = 1$.
    \item $F_n(x)$ 在每个观测值 $x_{(i)}$ 处是右连续的,点 $x_{(i)}$ 是 $F_n(x)$ 的跳跃间断点,$F_n(x)$ 在该点的跃度就是频率 $f_i \, (i=1,2,\cdots,k)$.
\end{enumerate}

对于给定的实数 $x$,当给出总体 $X$ 的不同的样本观测值时,相应的样本分布函数 $F_n(x)$ 的值有可能是不同的,因此 $F_n(x)$ 是一个随机变量.当给定样本观测值 $x_1,x_2,\cdots,x_n$ 时,$F_n(x)$ 是在 $n$ 次独立重复试验中事件 $\{ X \leqslant x \}$ 发生的频率.由于总体 $X$ 的分布函数是事件 $\{ X \leqslant x \}$ 发生的概率,根据伯努利定理可知,当 $n \to \infty$ 时,对于任意给定的正数 $\varepsilon$,有
$$
\lim_{n \to \infty} P \{ |F_n(x) - F(x)| < \varepsilon \} = 1
$$

\begin{theorem}[][格利文科定理]
    当 $n \to \infty$ 时,样本分布函数 $F_n(x)$ 依概率1关于 $x$ 一致收敛于总体分布函数 $F(x)$,即
    $$
    P \{ \lim_{n \to \infty} \sup_{-\infty < x < +\infty} |F_n(x) - F(x)| = 0 \} = 1
    $$
\end{theorem}

\section{样本函数及其概率分布}

\begin{definition}
    设 $X_1,X_2,\cdots,X_n$ 是来自总体 $X$ 的样本,$x_1,x_2,\cdots,x_n$ 是样本观测值.如果 $g(t_1,t_2,\cdots,t_n)$ 为已知的 $n$ 元函数,则称 $g(X_1,X_2,\cdots,X_n)$ 为\textbf{样本函数},它是一个随机变量,称 $g(x_1,x_2,\cdots,x_n)$ 为\textbf{样本函数的观测值}.如果样本函数 $g(X_1,X_2,\cdots,X_n)$ 中不含有未知参数,则称这种样本函数为\textbf{统计量}.
\end{definition}

常用的统计量:

\begin{enumerate}
    \item \textbf{样本均值}:$\overline{X} = \dfrac{1}{n} \displaystyle\sum_{i=1}^n X_i$ \\[0.5em]
    观测值:$\overline{x} = \dfrac{1}{n} \displaystyle\sum_{i=1}^n x_i$\\[0.5em]
    设总体 $X$ 具有数学期望 $E(X) = \mu$ 和方差 $D(X) = \sigma^2 (\sigma > 0)$,则 $E(X_i) = \mu$,$D(X_i) = \sigma^2 \, (i=1,2,\cdots,n)$,从而有
    $$
    \begin{aligned}
        & E(\overline{X}) = E(\dfrac{1}{n} \sum_{i=1}^n X_i) = \dfrac{1}{n} \sum_{i=1}^n E(X_i) = \mu \\
        & D(\overline{X}) = D(\dfrac{1}{n} \sum_{i=1}^n X_i) = \dfrac{1}{n^2} \sum_{i=1}^n D(X_i) = \dfrac{\sigma^2}{n}
    \end{aligned}
    $$

    \item \textbf{样本方差}:$S^2 = \dfrac{1}{n-1} \displaystyle\sum_{i=1}^n (X_i - \overline{X})^2 = \dfrac{1}{n-1} \left( \displaystyle\sum_{i=1}^n X_i^2 - n \overline{X}^2 \right)$ \\[0.5em]
    观测值:$s^2 = \dfrac{1}{n-1} \displaystyle\sum_{i=1}^n (x_i - \overline{x})^2 = \dfrac{1}{n-1} \left( \displaystyle\sum_{i=1}^n x_i^2 - n \overline{x}^2 \right)$ \\[0.5em]
    设总体 $X$ 具有数学期望 $E(X) = \mu$ 和方差 $D(X) = \sigma^2 (\sigma > 0)$,则
    $$
    \begin{aligned}
        E(S^2) &= E \left( \dfrac{1}{n-1} \left( \displaystyle\sum_{i=1}^n X_i^2 - n \overline{X}^2 \right) \right) \\
        &= \dfrac{1}{n-1} \left[ \displaystyle\sum_{i=1}^n E(X_i^2) - nE(\overline{X}^2) \right] \\
        &= \dfrac{1}{n-1} \left\{ \sum_{i=1}^n [D(X_i) + (E(X_i))^2] - n [D(\overline{X}) + (E(\overline{X}))^2] \right\} \\
        &= \dfrac{1}{n-1} \left( n \sigma^2 + n \mu^2 - n \cdot \dfrac{\sigma^2}{n} - n \mu^2 \right) \\
        &= \sigma^2
    \end{aligned}
    $$

    \item \textbf{样本标准差}:$S = \sqrt{S^2} = \sqrt{\dfrac{1}{n-1} \displaystyle\sum_{i=1}^n (X_i - \overline{X})^2}$ \\[0.5em]
    观测值:$s = \sqrt{s^2} = \sqrt{\dfrac{1}{n-1} \displaystyle\sum_{i=1}^n (x_i - \overline{x})^2}$

    \item \textbf{样本} $k$ \textbf{阶原点矩}:$A_k = \dfrac{1}{n} \displaystyle\sum_{i=1}^n X_i^k, \; k=1,2,\cdots$ \\[0.5em]
    观测值:$a_k = \dfrac{1}{n} \displaystyle\sum_{i=1}^n x_i^k, \; k=1,2,\cdots$ \\[0.5em]
    样本一阶原点矩就是样本均值,即
    $$
    A_1 = \overline{X}
    $$

    \item \textbf{样本} $k$ \textbf{阶中心矩}:$B_k = \dfrac{1}{n} \displaystyle\sum_{i=1}^n (X_i - \overline{X})^k, k=1,2,\cdots$ \\[0.5em]
    观测值:$b_k = \dfrac{1}{n} \displaystyle\sum_{i=1}^n (x_i - \overline{x})^k, k=1,2,\cdots$ \\[0.5em]
    样本一阶中心矩等于零,即 $B_1 = 0$. \\
    样本二阶中心矩 $B_2$ 和样本方差 $S^2$ 有如下关系:$B_2 = \dfrac{n-1}{n} S^2$\\
    如果总体 $X$ 具有方差 $\sigma^2$,则有
    $$
    E(B_2) = E(\dfrac{n-1}{n} S^2) = \dfrac{n-1}{n} E(S^2) = \dfrac{n-1}{n} \sigma^2
    $$

    \item 样本最大值和样本最小值 \\
    设 $X_1, X_2, \cdots, X_n$ 是来自总体 $X$ 的样本,对于每一样本观测值 $x_1, x_2, \cdots, x_n$,取
    $$
    \begin{aligned}
        & x_{(n)} = \max(x_1, x_2, \cdots, x_n) \\
        & x_{(1)} = \min(x_1, x_2, \cdots, x_n)
    \end{aligned}
    $$
    分别作为随机变量 $X_{(n)}$ 和 $X_{(1)}$ 的观测值,称 $X_{(n)}$ 和 $X_{(1)}$ 分别为\textbf{样本最大值}和\textbf{样本最小值},分别记为
    $$
    \begin{aligned}
        & X_{(n)} = \max(X_1, X_2, \cdots, X_n) \\
        & X_{(1)} = \min(X_1, X_2, \cdots, X_n)
    \end{aligned}
    $$
    设总体 $X$ 的分布函数为 $F(x)$,记 $X_{(n)}$ 和 $X_{(1)}$ 的分布函数依次为 $F_{\text{max}}(x)$ 和 $F_{\text{min}}(x)$ ,则
    $$
    \begin{aligned}
        F_{\text{max}}(x) &= P \{ \max(X_1, X_2, \cdots, X_n) \leqslant x \} \\
        &= P \{ X_1 \leqslant x, X_2 \leqslant x, \cdots, X_n \leqslant x \} \\
        &= P \{ X_1 \leqslant x \} \, P \{ X_2 \leqslant x \} \cdots P \{ X_n \leqslant x \} \\
        &= [F(x)]^n \\
        \\
        F_{\text{min}}(x) &= P \{ \min(X_1, X_2, \cdots, X_n) \leqslant x \} \\
        &= 1 - P \{ \min(X_1, X_2, \cdots, X_n) > x \} \\
        &= 1 - P \{ X_1 > x \} \, P \{ X_2 > x \} \cdots P \{ X_n > x \} \\
        &= 1-[1-F(x)]^n
    \end{aligned}
    $$
\end{enumerate}

\begin{theorem} \label{theorem:正态总体的样本均值服从正态分布}
    设 $X \sim N(\mu, \sigma^2)$,$X_1, X_2, \cdots, X_n$ 是来自总体 $X$ 的样本,$\overline{X}$ 为样本均值,则随机变量
    $$
    u = \dfrac{\overline{X} - \mu}{\sigma / \sqrt{n}} \sim N(0,1)
    $$
\end{theorem}

\begin{proof}
    由于 $X_1, X_2, \cdots, X_n$ 相互独立,$X_i \sim N(\mu, \sigma^2) \, (i=1,2,\cdots,n)$,因此 $\overline{X} = \dfrac{1}{n} \displaystyle\sum_{i=1}^n X_i$ 服从正态分布.而 $E(\overline{X}) = \mu$,$D(\overline{X}) = \dfrac{\sigma^2}{n}$,故
    $$
    \overline{X} \sim N(\mu, \dfrac{\sigma^2}{n})
    $$
    将随机变量 $\overline{X}$ 标准化,可得
    $$
    u = \dfrac{\overline{X} - \mu}{\sigma / \sqrt{n}} \sim N(0,1)
    $$
\end{proof}

\begin{theorem} \label{theorem:两个正太总体的样本均值}
    设 $X \sim N(\mu_1, \sigma_1^2)$,$Y \sim N(\mu_2, \sigma_2^2)$,分别独立地从总体 $X$ 和总体 $Y$ 中抽取样本 $X_1, X_2, \cdots, X_{n_1}$ 及 $Y_1, Y_2, \cdots, Y_{n_2}$,样本均值分别为 $\overline{X}$ 和 $\overline{Y}$,则随机变量
    $$
    u = \dfrac{\overline{X} - \overline{Y} - (\mu_1 - \mu_2)}{\sqrt{\dfrac{\sigma_1^2}{n_1} + \dfrac{\sigma_2^2}{n_2}}} \sim N(0,1)
    $$
\end{theorem}

\begin{proof}
    样本 $X_1, X_2, \cdots, X_{n_1}$ 和 $Y_1, Y_2, \cdots, Y_{n_2}$ 相互独立,也就是两个多维随机变量 $(X_1, X_2, \cdots, X_{n_1})$ 与 $(Y_1, Y_2, \cdots, Y_{n_2})$ 相互独立,因此 $\overline{X}$ 和 $\overline{Y}$ 相互独立,且
    $$
    \overline{X} \sim N(\mu_1, \dfrac{\sigma_1^2}{n_1}), \quad \overline{Y} \sim N(\mu_2, \dfrac{\sigma_2^2}{n_2})
    $$
    从而有
    $$
    \overline{X} - \overline{Y} \sim N(\mu_1 - \mu_2, \dfrac{\sigma_1^2}{n_1} + \dfrac{\sigma_2^2}{n_2})
    $$
    所以
    $$
    u = \dfrac{\overline{X} - \overline{Y} - (\mu_1 - \mu_2)}{\sqrt{\dfrac{\sigma_1^2}{n_1} + \dfrac{\sigma_2^2}{n_2}}} \sim N(0,1)
    $$
\end{proof}

\section{来自正态总体的统计量及其分布}

\subsection{\texorpdfstring{$\chi^2$}{} 分布}

\begin{definition} \label{def:卡方分布}
    设 $X_1, X_2, \cdots, X_n$ 是来自标准正态总体 $N(0,1)$ 的样本,称统计量
    $$
    \chi^2 = X_1^2 + X_2^2 + \cdots + X_n^2
    $$
    服从自由度为 $n$ 的 $\chi^2$ 分布,记作 $\chi^2 \sim \chi^2(n)$.
\end{definition}

若 $\chi^2 \sim \chi^2(n)$,则 $\chi^2$ 的概率密度为
$$
f(x) = \begin{cases}
    \dfrac{1}{2^{\frac{n}{2}} \Gamma(\frac{n}{2})} x^{\frac{n}{2} - 1} e^{-\frac{x}{2}} & x>0 \\
    0 & x \leqslant 0
\end{cases}
$$

\begin{property} \label{prop:卡方分布的数学期望和方差}
    若 $\chi^2 \sim \chi^2(n)$,则 $E(\chi^2) = n$,$D(\chi^2) = 2n$.
\end{property}

\begin{proof}
    $\chi^2 = \displaystyle\sum_{i=1}^n X_i^2$,其中 $X_i \sim N(0,1) \, (i=1,2,\cdots,n)$,$X_1, X_2, \cdots, X_n$ 相互独立,且 $E(X_i) = 0$,$D(X_i) = 1$,从而有
    $$
    E(X_i^2) = E([X_i - E(X_i)]^2) = D(X_i) = 1
    $$
    因此
    $$
    E(\chi^2) = \sum_{i=1}^n E(X_i^2) = n
    $$
    因为
    $$
    E(X_i^4) = \int_{-\infty}^{+\infty} x^4 f(x) \, \text{d}x = \dfrac{1}{\sqrt{2 \pi}} \int_{-\infty}^{+\infty} x^4 e^{-\frac{x^2}{2}} \text{d}x = 3
    $$
    所以
    $$
    D(X_i^2) = E(X_i^4) - [E(X_i^2)]^2 = 3-1 = 2
    $$
    从而有
    $$
    D(\chi^2) = \sum_{i=1}^n D(X_i^2) = 2n
    $$
\end{proof}

\begin{property}[][可加性]
    设 $X \sim \chi^2(n_1)$,$Y \sim \chi^2(n_2)$,且 $X$ 与 $Y$ 相互独立,则 $X+Y \sim \chi^2(n_1 + n_2)$.
\end{property}

\begin{property} \label{prop:卡方分布趋向于正态分布}
    设 $\chi^2 \sim \chi^2(n)$,则对任意实数 $x$,有
    $$
    \lim_{n \to \infty} P \left\{ \dfrac{\chi^2 - n}{\sqrt{2n}} \leqslant x \right\} = \dfrac{1}{\sqrt{2 \pi}} \int_{-\infty}^{x} e^{-\frac{t^2}{2}} \text{d}t
    $$
\end{property}

性质\ref*{prop:卡方分布趋向于正态分布} 说明当 $n$ 很大时,$\dfrac{\chi^2 - n}{\sqrt{2n}}$ 近似服从标准正态分布,也即自由度为 $n$ 的 $\chi^2$ 分布近似于正态分布 $N(n,2n)$.

\begin{definition}
    设 $\chi^2 \sim \chi^2(n)$,$f(x)$ 为其概率密度,对于给定的正数 $\alpha \, (0 < \alpha < 1)$,称满足条件
    $$
    P \{ \chi^2 > \chi_{\alpha}^2(n) \} = \int_{\chi_{\alpha}^2(n)}^{+\infty} f(x) \, \text{d}x = \alpha
    $$
    的点 $\chi_{\alpha}^2(n)$ 为 $\chi^2$ 分布的\textbf{上} $\alpha$ \textbf{分位点(数)}.
\end{definition}

\begin{theorem} \label{theorem:来自正态总体的样本标准化后求和,服从卡方分布}
    设 $X_1,X_2,\cdots,X_n$ 是来自总体 $N(\mu,\sigma^2)$ 的样本,则随机变量
    $$
    \chi^2 = \dfrac{1}{\sigma^2} \sum_{i=1}^n (X_i - \mu)^2 \sim \chi^2(n)
    $$
\end{theorem}

\begin{proof}
    随机变量 $X_1,X_2,\cdots,X_n$ 相互独立,且都与总体服从相同的分布 $N(\mu,\sigma^2)$,将 $X_i$ 标准化,得
    $$
    \dfrac{X_i - \mu}{\sigma} \sim N(0,1), \; i=1,2,\cdots,n
    $$
    且 $\dfrac{X_1 - \mu}{\sigma}, \dfrac{X_2 - \mu}{\sigma},\cdots, \dfrac{X_n - \mu}{\sigma}$ 相互独立.根据定义 \ref{def:卡方分布} 可得
    $$
    \chi^2 = \sum_{i=1}^n \left( \dfrac{X_i - \mu}{\sigma} \right)^2 = \dfrac{1}{\sigma^2} \sum_{i=1}^n (X_i - \mu)^2 \sim \chi^2(n)
    $$
\end{proof}

\begin{theorem} \label{theorem:正态总体的样本均值与样本方差相互独立}
    设总体 $X$ 服从正态分布 $N(\mu,\sigma^2)$,从总体 $X$ 中抽取样本 $X_1,X_2,\cdots,X_n$,样本均值和样本方差分别为 $\overline{X}$ 和 $S^2$,则

    \begin{enumerate}
        \item $\overline{X}$ 与 $S^2$ 相互独立. \vspace{0.5em}
        \item $\dfrac{(n-1) S^2}{\sigma^2} \sim \chi^2(n-1)$
    \end{enumerate}
\end{theorem}

\vspace{-1.8em}

\begin{proof}
    设有 $n$ 维正交矩阵 $\boldsymbol{A}$,且其第一行元素均为 $\dfrac{1}{\sqrt{n}}$.记
    $$
    (Y_1, Y_2, \cdots, Y_n)^{\text{T}} = \boldsymbol{A} (X_1, X_2, \cdots, X_n)^{\text{T}}
    $$
    则
    $$
    \begin{aligned}
        \sum_{i=1}^n Y_i^2 &= (Y_1, Y_2, \cdots, Y_n) (Y_1, Y_2, \cdots, Y_n)^{\text{T}} \\
        &= [\boldsymbol{A} (X_1, X_2, \cdots, X_n)^{\text{T}}]^{\text{T}} [\boldsymbol{A} (X_1, X_2, \cdots, X_n)^{\text{T}}] \\
        &= (X_1, X_2, \cdots, X_n) \boldsymbol{A}^{\text{T}} \boldsymbol{A} (X_1, X_2, \cdots, X_n)^{\text{T}} \\
        &= (X_1, X_2, \cdots, X_n) \boldsymbol{E} (X_1, X_2, \cdots, X_n)^{\text{T}} \\
        &= \sum_{i=1}^n X_i^2
    \end{aligned}
    $$
    又由
    $$
    \begin{aligned}
        Y_1 &= \dfrac{1}{\sqrt{n}} X_1 + \dfrac{1}{\sqrt{n}} X_2 + \cdots + \dfrac{1}{\sqrt{n}} X_n \\
        &= \dfrac{1}{\sqrt{n}} \sum_{i=1}^{n} X_i \\
        &= \sqrt{n} \cdot \dfrac{1}{n} \sum_{i=1}^{n} X_i \\
        &= \sqrt{n} \overline{X}
    \end{aligned}
    $$
    则
    $$
    \overline{X} = \dfrac{1}{\sqrt{n}} Y_1
    $$
    进一步有
    $$
    \begin{aligned}
        S^2 &= \dfrac{1}{n-1} \left( \sum_{i=1}^n X_i^2 - n \overline{X}^2 \right) \\
        &= \dfrac{1}{n-1} \left( \sum_{i=1}^n Y_i^2 - n \dfrac{1}{n} Y_1^2 \right) \\
        &= \dfrac{1}{n-1} \sum_{i=2}^n Y_i^2
    \end{aligned}
    $$
    由于 $X_1,X_2,\cdots,X_n$ 独立同分布,则 $n$ 维随机变量 $(X_1, X_2, \cdots, X_n)$ 的联合概率密度为
    $$
    \begin{aligned}
        f(x_1, x_2, \cdots, x_n) &= \prod_{i=1}^n f_{X_i}(x_i) \\
        &= \prod_{i=1}^n \dfrac{1}{\sqrt{2 \pi} \sigma} e^{-\frac{(x_i - \mu)^2}{2 \sigma^2}} \\
        &= \dfrac{1}{(\sqrt{2 \pi} \sigma)^n} \exp{ \left[-\frac{1}{2 \sigma^2} \sum_{i=1}^{n} (x_i - \mu)^2 \right] } \\
        &= \dfrac{1}{(\sqrt{2 \pi} \sigma)^n} \exp{ \left[-\frac{1}{2 \sigma^2} \left( \sum_{i=1}^{n} x_i^2 - 2 \mu \sum_{i=1}^{n} x_i + n \mu^2 \right) \right] } \\
        &= \dfrac{1}{(\sqrt{2 \pi} \sigma)^n} \exp{ \left[-\frac{1}{2 \sigma^2} \left( \sum_{i=1}^{n} x_i^2 - 2 \mu n \overline{x} + n \mu^2 \right) \right] }
    \end{aligned}
    $$
    由 $Y_1, Y_2, \cdots, Y_n$ 与 $X_1,X_2,\cdots,X_n$ 之间的代换关系,可得 $(Y_1, Y_2, \cdots, Y_n)$ 的联合概率密度为
    $$
    \begin{aligned}
        f(y_1, y_2, \cdots, y_n) &= f(g_1(x_1, x_2, \cdots, x_n), g_2(x_1, x_2, \cdots, x_n), \cdots, g_n(x_1, x_2, \cdots, x_n)) |J| \\
        &= |J| \dfrac{1}{(\sqrt{2 \pi} \sigma)^n} \exp{ \left[-\frac{1}{2 \sigma^2} \left( \sum_{i=1}^{n} y_i^2 - 2 \mu n \dfrac{1}{\sqrt{n}} y_1 + n \mu^2 \right) \right] } \\
        &= |J| \dfrac{1}{(\sqrt{2 \pi} \sigma)^n} \exp{ \left[-\frac{1}{2 \sigma^2} \left( \sum_{i=1}^{n} y_i^2 - 2 \mu \sqrt{n} y_1 + n \mu^2 \right) \right] } \\
        &= |J| \dfrac{1}{\sqrt{2 \pi} \sigma} e^{-\frac{1}{2 \sigma^2} (y_1^2 - 2 \mu \sqrt{n} y_1 + n \mu^2)} \times \dfrac{1}{\sqrt{2 \pi} \sigma} e^{-\frac{1}{2 \sigma^2} y_2^2} \times \cdots \times \dfrac{1}{\sqrt{2 \pi} \sigma} e^{-\frac{1}{2 \sigma^2} y_n^2} \\
        &= |J| \dfrac{1}{\sqrt{2 \pi} \sigma} e^{-\frac{1}{2 \sigma^2} (y_1 - \sqrt{n} \mu)^2} \times \dfrac{1}{\sqrt{2 \pi} \sigma} e^{-\frac{1}{2 \sigma^2} y_2^2} \times \cdots \times \dfrac{1}{\sqrt{2 \pi} \sigma} e^{-\frac{1}{2 \sigma^2} y_n^2} \\
    \end{aligned}
    $$
    由于 $\boldsymbol{A}$ 为正交矩阵,可得 $|J|=1$,所以
    $$
    f(y_1, y_2, \cdots, y_n) = \dfrac{1}{\sqrt{2 \pi} \sigma} e^{-\frac{1}{2 \sigma^2} (y_1 - \sqrt{n} \mu)^2} \times \dfrac{1}{\sqrt{2 \pi} \sigma} e^{-\frac{1}{2 \sigma^2} y_2^2} \times \cdots \times \dfrac{1}{\sqrt{2 \pi} \sigma} e^{-\frac{1}{2 \sigma^2} y_n^2}
    $$
    因此 $Y_1, Y_2, \cdots, Y_n$ 相互独立,且 $Y_1 \sim N(\sqrt{n} \mu, \sigma^2)$,$Y_2, Y_3, \cdots, Y_n \sim N(0, \sigma^2)$.由于 $\overline{X} = \dfrac{1}{\sqrt{n}} Y_1$,$S^2 = \dfrac{1}{n-1} \displaystyle\sum_{i=2}^n Y_i^2$,所以 $\overline{X}$ 与 $S^2$ 相互独立.

    因为 $S^2 = \dfrac{1}{n-1} \displaystyle\sum_{i=2}^n Y_i^2$,由定理 \ref{theorem:来自正态总体的样本标准化后求和,服从卡方分布} 可得
    $$
    \dfrac{(n-1) S^2}{\sigma^2} = \dfrac{1}{\sigma^2} \sum_{i=2}^n Y_i^2 \sim \chi^2(n-1)
    $$
\end{proof}

\begin{theorem}
    设总体 $X \sim N(\mu,\sigma^2)$,从总体 $X$ 中抽取样本容量为 $n$ 的样本,其样本方差为 $S^2$,则
    $$
    D(S^2) = \dfrac{2 \sigma^4}{n-1}
    $$
\end{theorem}

\begin{proof}
    由定理 \ref{theorem:正态总体的样本均值与样本方差相互独立} 可知
    $$
    \dfrac{(n-1) S^2}{\sigma^2} \sim \chi^2(n-1)
    $$
    由 $\chi^2$ 分布的性质 \ref*{prop:卡方分布的数学期望和方差} 可得
    $$
    D \left( \dfrac{(n-1) S^2}{\sigma^2} \right) = 2(n-1)
    $$
    因为
    $$
    D \left( \dfrac{(n-1) S^2}{\sigma^2} \right) = \dfrac{(n-1)^2}{\sigma^4} D(S^2)
    $$
    所以
    $$
    D(S^2) = \dfrac{2 \sigma^4}{n-1}
    $$
\end{proof}

\subsection{\texorpdfstring{$t$}{} 分布}

\begin{definition} \label{def:t分布}
    设 $X \sim N(0,1)$,$Y \sim \chi^2(n)$,且 $X$ 与 $Y$ 相互独立,称随机变量
    $$
    t = \dfrac{X}{\sqrt{Y/n}}
    $$
    服从\textbf{自由度为} $n$ \textbf{的} $t$ \textbf{分布},记作 $t \sim t(n)$.
\end{definition}

$t(n)$ 分布的概率密度为
$$
f(x) = \dfrac{\Gamma(\frac{n+1}{2})}{\sqrt{n \pi} \, \Gamma(\frac{n}{2})} \left( 1 + \dfrac{x^2}{n} \right)^{-\frac{n+1}{2}}, \; -\infty < x < +\infty
$$

$f(x)$ 是偶函数,其图像关于 $y$ 轴对称.

当 $n \to \infty$ 时,$t(n)$ 分布的概率密度趋向于标准正态分布的概率密度,即
$$
\lim_{n \to \infty} f(x) = \dfrac{1}{\sqrt{2 \pi}} e^{-\frac{x^2}{2}}, \; -\infty < x < +\infty
$$
因此,当自由度 $n$ 很大时,$t(n)$ 分布近似于标准正态分布.

\begin{definition}
    设 $t \sim t(n)$,其概率密度为 $f(x)$,对于给定的正数 $\alpha \, (0 < \alpha < 1)$,称满足条件
    $$
    P \{ t > t_{\alpha}(n) \} = \int_{t_{\alpha}(n)}^{+\infty} f(x) \, \text{d}x = \alpha
    $$
    的点 $t_{\alpha}(n)$ 为 $t(n)$ 分布的\textbf{上} $\alpha$ \textbf{分位点(数)}.
\end{definition}

根据 $t(n)$ 分布概率密度曲线的对称性可知
$$
t_{1 - \alpha}(n) = -t_{\alpha}(n)
$$

当 $n$ 很大($n \geqslant 50$)时,$t_{\alpha}(n)$ 近似等于标准正态分布的上 $\alpha$ 分位点 $u_{\alpha}$,即 $t_{\alpha}(n) \approx u_{\alpha}$.

\begin{theorem}
    设总体 $X$ 服从正态分布 $N(\mu,\sigma^2)$,从总体 $X$ 中抽取样本 $X_1,X_2,\cdots,X_n$,样本均值和样本方差分别为 $\overline{X}$ 和 $S^2$,则随机变量
    $$
    t = \dfrac{\overline{X} - \mu}{S} \sqrt{n} \sim t(n-1)
    $$
\end{theorem}

\begin{proof}
    由定理 \ref{theorem:正态总体的样本均值服从正态分布} 可知
    $$
    u = \dfrac{\overline{X} - \mu}{\sigma} \sqrt{n} \sim N(0,1)
    $$
    由定理 \ref{theorem:正态总体的样本均值与样本方差相互独立} 可知 $\overline{X}$ 与 $S^2$ 相互独立,并且
    $$
    \chi^2 = \dfrac{(n-1) S^2}{\sigma^2} \sim \chi^2(n-1)
    $$
    从而可得 $u$ 与 $\chi^2$ 相互独立.根据定义 \ref{def:t分布} 可得
    $$
    t = \dfrac{u}{\sqrt{\dfrac{\chi^2}{n-1}}} = \dfrac{\dfrac{\overline{X} - \mu}{\sigma} \sqrt{n}}{\sqrt{\dfrac{(n-1) S^2}{\sigma^2 (n-1)}}} = \dfrac{\overline{X} - \mu}{S} \sqrt{n} \sim t(n-1)
    $$
\end{proof}

\begin{theorem}
    若从两个正态总体 $N(\mu_1, \sigma^2)$ 和 $N(\mu_2, \sigma^2)$ 中分别独立地抽取样本,样本容量依次为 $n_1$ 和 $n_2$,样本均值依次为 $\overline{X}$ 和 $\overline{Y}$,样本方差依次为 $S_1^2$ 和 $S_2^2$.记
    $$
    S_W = \sqrt{\dfrac{(n_1 - 1) S_1^2 + (n_2 - 1) S_2^2}{n_1 + n_2 - 2}}
    $$
    则随机变量
    $$
    t = \dfrac{\overline{X} - \overline{Y} - (\mu_1 - \mu_2)}{S_W \sqrt{\dfrac{1}{n_1} + \dfrac{1}{n_2}}} \sim t(n_1 + n_2 - 2)
    $$
\end{theorem}

\begin{proof}
    根据定理 \ref{theorem:两个正太总体的样本均值} 可得
    $$
    u = \dfrac{\overline{X} - \overline{Y} - (\mu_1 - \mu_2)}{\sigma \sqrt{\dfrac{1}{n_1} + \dfrac{1}{n_2}}} \sim N(0,1)
    $$
    根据定理 \ref{theorem:正态总体的样本均值与样本方差相互独立} 可知
    $$
    \begin{aligned}
        & \chi_1^2 = \dfrac{(n_1 - 1) S_1^2}{\sigma^2} \sim \chi^2(n_1 - 1) \\
        & \chi_2^2 = \dfrac{(n_2 - 1) S_2^2}{\sigma^2} \sim \chi^2(n_2 - 1)
    \end{aligned}
    $$
    因为两个样本相互独立,所以 $S_1^2$ 与 $S_2^2$ 相互独立,从而可知 $\chi_1^2$ 与 $\chi_2^2$ 相互独立.由 $\chi^2$ 分布的可加性得
    $$
    \chi_1^2 + \chi_2^2 = \dfrac{(n_1 - 1) S_1^2 + (n_2 - 1) S_2^2}{\sigma^2} \sim \chi^2(n_1 + n_2 - 2)
    $$
    根据定理 \ref{theorem:正态总体的样本均值与样本方差相互独立} 可得 $\overline{X}$ 与 $S_1^2$ 相互独立,$\overline{Y}$ 与 $S_2^2$ 相互独立,因此 $u$ 与 $\chi_1^2 + \chi_2^2$ 相互独立.由定义 \ref{def:t分布} 可知
    $$
    t = \dfrac{u}{\sqrt{\dfrac{\chi_1^2 + \chi_2^2}{n_1 + n_2 - 2}}} = \dfrac{\dfrac{\overline{X} - \overline{Y} - (\mu_1 - \mu_2)}{\sigma \sqrt{\dfrac{1}{n_1} + \dfrac{1}{n_2}}}}{\sqrt{\dfrac{(n_1 - 1) S_1^2 + (n_2 - 1) S_2^2}{\sigma^2 (n_1 + n_2 - 2)}}} = \dfrac{\overline{X} - \overline{Y} - (\mu_1 - \mu_2)}{S_W \sqrt{\dfrac{1}{n_1} + \dfrac{1}{n_2}}} \sim t(n_1 + n_2 - 2)
    $$
\end{proof}

\begin{conclusion}
    设随机变量 $X$ 与 $Y$ 相互独立,且 $X \sim N(\mu, \sigma^2)$,$\dfrac{Y}{\sigma^2} \sim \chi^2(n)$,则
    $$
    t = \dfrac{X - \mu}{\sqrt{Y / n}} \sim t(n)
    $$
\end{conclusion}

\begin{proof}
    由 $X \sim N(\mu, \sigma^2)$ 可得 $\dfrac{X - \mu}{\sigma} \sim N(0,1)$.根据定义 \ref{def:t分布} 可得
    $$
    t = \dfrac{\dfrac{X - \mu}{\sigma}}{\sqrt{\dfrac{Y}{n \sigma^2}}} = \dfrac{X - \mu}{\sqrt{Y / n}} \sim t(n)
    $$
\end{proof}

\subsection{\texorpdfstring{$F$}{} 分布}

\begin{definition} \label{def:F分布}
    设 $X \sim \chi^2(n_1)$,$Y \sim \chi^2(n_2)$,且 $X$ 与 $Y$ 相互独立,称随机变量
    $$
    F = \dfrac{X / n_1}{Y / n_2}
    $$
    服从第一自由度为 $n_1$、第二自由度为 $n_2$(或自由度为 $(n_1,n_2)$)的 $F$ \textbf{分布},记作
    $$
    F \sim F(n_1, n_2)
    $$
\end{definition}

$F(n_1, n_2)$ 分布的概率密度为
$$
f(x) = \begin{cases}
    \dfrac{\Gamma(\frac{n_1 + n_2}{2})}{\Gamma(\frac{n_1}{2}) \Gamma(\frac{n_2}{2})} \left( \dfrac{n_1}{n_2} \right)^{\frac{n_1}{2}} x^{\frac{n_1}{2} - 1} \left( 1 + \dfrac{n_1}{n_2} x \right)^{-\frac{n_1 + n_2}{2}} & x>0 \\[0.5em]
    0 & x \leqslant 0
\end{cases}
$$

若 $F \sim F(n_1, n_2)$,则 $\dfrac{1}{F} \sim F(n_2, n_1)$.

\begin{definition}
    设 $F \sim F(n_1, n_2)$,其概率密度为 $f(x)$,对于给定的正数 $\alpha \, (0 < \alpha < 1)$,称满足条件
    $$
    P \{ F > F_{\alpha}(n_1, n_2) \} = \int_{F_{\alpha}(n_1, n_2)}^{+\infty} f(x) \, \text{d}x = \alpha
    $$
    的点 $F_{\alpha}(n_1, n_2)$ 为 $F(n_1, n_2)$ 分布的\textbf{上} $\alpha$ \textbf{分位点(数)}.
\end{definition}

\begin{property}
    $F_{1 - \alpha}(n_1, n_2) = \dfrac{1}{F_{\alpha}(n_2, n_1)}$
\end{property}

\begin{proof}
    设 $F \sim F(n_1, n_2)$,则对于 $\alpha \, (0 < \alpha < 1)$,有
    $$
    \begin{aligned}
        1 - \alpha &= P \{ F > F_{1 - \alpha}(n_1, n_2) \} \\
        &= P \left\{ \dfrac{1}{F} < \dfrac{1}{F_{1 - \alpha}(n_1, n_2)} \right\} \\
        &= 1 - P \left\{ \dfrac{1}{F} \geqslant \dfrac{1}{F_{1 - \alpha}(n_1, n_2)} \right\} \\
        &= 1 - P \left\{ \dfrac{1}{F} > \dfrac{1}{F_{1 - \alpha}(n_1, n_2)} \right\}
    \end{aligned}
    $$
    于是有
    $$
    P \left\{ \dfrac{1}{F} > \dfrac{1}{F_{1 - \alpha}(n_1, n_2)} \right\} = \alpha
    $$
    由于 $\dfrac{1}{F} \sim F(n_2, n_1)$,因此 $\dfrac{1}{F_{1 - \alpha}(n_1, n_2)}$ 就是 $F(n_2, n_1)$ 的上 $\alpha$ 分位点,即
    $$
    F_{\alpha}(n_2, n_1) = \dfrac{1}{F_{1 - \alpha}(n_1, n_2)}
    $$
    所以
    $$
    F_{1 - \alpha}(n_1, n_2) = \dfrac{1}{F_{\alpha}(n_2, n_1)}
    $$
\end{proof}

\begin{theorem}
    设 $X_1, X_2, \cdots, X_{n_1}$ 是来自正态总体 $N(\mu_1, \sigma_1^2)$ 的样本,$Y_1, Y_2, \cdots, Y_{n_2}$ 是来自正态总体 $N(\mu_2, \sigma_2^2)$ 的样本,并且这两个样本相互独立,则随机变量
    $$
    F = \dfrac{n_2}{n_1} \cdot \dfrac{\sigma_2^2}{\sigma_1^2} \cdot \dfrac{\displaystyle\sum_{i=1}^{n_1} (X_i - \mu_1)^2}{\displaystyle\sum_{j=1}^{n_2} (Y_j - \mu_2)^2} \sim F(n_1, n_2)
    $$
\end{theorem}

\begin{proof}
    根据定理 \ref{theorem:来自正态总体的样本标准化后求和,服从卡方分布} 可得
    $$
    \begin{aligned}
        & \chi_1^2 = \dfrac{1}{\sigma^2} \sum_{i=1}^{n_1} (X_i - \mu_1)^2 \sim \chi^2(n_1) \\
        & \chi_2^2 = \dfrac{1}{\sigma^2} \sum_{j=1}^{n_2} (Y_i - \mu_2)^2 \sim \chi^2(n_2)
    \end{aligned}
    $$
    因为两个样本相互独立,所以 $\chi_1^2$ 与 $\chi_2^2$ 相互独立.由定义 \ref{def:F分布} 可得
    $$
    F = \dfrac{\chi_1^2 / n_1}{\chi_2^2 / n_2} = \dfrac{n_2}{n_1} \cdot \dfrac{\sigma_2^2}{\sigma_1^2} \cdot \dfrac{\displaystyle\sum_{i=1}^{n_1} (X_i - \mu_1)^2}{\displaystyle\sum_{j=1}^{n_2} (Y_j - \mu_2)^2} \sim F(n_1, n_2)
    $$
\end{proof}

\begin{theorem}
    若从两个正态总体 $N(\mu_1, \sigma^2)$ 和 $N(\mu_2, \sigma^2)$ 中分别独立地抽取样本,样本容量依次为 $n_1$ 和 $n_2$,样本均值依次为 $\overline{X}$ 和 $\overline{Y}$,样本方差依次为 $S_1^2$ 和 $S_2^2$,则随机变量
    $$
    F = \dfrac{\sigma_2^2}{\sigma_1^2} \cdot \dfrac{S_1^2}{S_2^2} \sim F(n_1 - 1, n_2 - 1)
    $$
\end{theorem}

\begin{proof}
    根据定理 \ref{theorem:正态总体的样本均值与样本方差相互独立} 可得
    $$
    \begin{aligned}
        & \chi_1^2 = \dfrac{(n_1 - 1) S_1^2}{\sigma_1^2} \sim \chi^2(n_1 - 1) \\
        & \chi_2^2 = \dfrac{(n_2 - 1) S_2^2}{\sigma_2^2} \sim \chi^2(n_2 - 1)
    \end{aligned}
    $$
    因为两个样本相互独立,所以 $S_1^2$ 和 $S_2^2$ 相互独立,从而可知 $\chi_1^2$ 与 $\chi_2^2$ 相互独立.由定义 \ref{def:F分布} 可得
    $$
    F = \dfrac{\chi_1^2 / (n_1 - 1)}{\chi_2^2 / (n_2 - 1)} = \dfrac{\sigma_2^2}{\sigma_1^2} \cdot \dfrac{S_1^2}{S_2^2} \sim F(n_1 - 1, n_2 - 1)
    $$
\end{proof}

\begin{conclusion}
    若 $X \sim t(n)$,则 $X^2 \sim F(1,n)$.
\end{conclusion}

\begin{proof}
    根据定义 \ref{def:t分布},若 $X \sim t(n)$,则存在 $Y \sim N(0,1)$,$Z \sim \chi^2(n)$,$Y$ 与 $Z$ 相互独立,且
    $$
    X = \dfrac{Y}{\sqrt{Z / n}}
    $$
    则
    $$
    X^2 = \dfrac{Y^2}{Z / n}
    $$
    根据定义 \ref{def:卡方分布} 可得 $Y^2 \sim \chi^2(1)$,则由定义 \ref{def:F分布} 可知 $X^2 \sim F(1,n)$.
\end{proof}