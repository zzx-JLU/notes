\chapter{二维随机变量及其分布}

\section{二维随机变量}

\subsection{二维随机变量及其分布函数}

\begin{definition}
    设随机试验 $E$ 的样本空间为 $\varOmega$,$X$ 和 $Y$ 是定义在 $\varOmega$ 上的两个随机变量,由它们构成的向量 $(X,Y)$ 称为\textbf{二维随机变量}或\textbf{二维随机向量}.
\end{definition}

\begin{definition}
    设 $(X,Y)$ 是二维随机变量.对于任意实数 $x$ 和 $y$,记事件 $\{X \leqslant x\}$ 与 $\{Y \leqslant y\}$ 的交事件为 $\{X \leqslant x, Y \leqslant y\}$,称二元函数
    $$
    F(x, y) = P\{X \leqslant x, Y \leqslant y\}, \; (x,y) \in \mathbf{R}^2
    $$
    为\textbf{二维随机变量} $(X,Y)$ \textbf{的分布函数},或称为随机变量 $X$ 和 $Y$ 的\textbf{联合分布函数}.
\end{definition}

如果将二维随机变量 $(X,Y)$ 看做 $xOy$ 平面上随机点的坐标,则分布函数 $F(x,y)$ 在点 $(x,y)$ 处的函数值就是随机点落在以 $(x,y)$ 为顶点且位于该点左下方的无界域内的概率,如图 \ref{fig:F(x,y)} 所示.

\begin{figure}[htbp]
    \centering

    \begin{tikzpicture}[>=Stealth, scale=0.55]
        % 坐标轴
        \draw[->] (-2, 0)--(4, 0) node[below]{$x$};
        \draw[->] (0, -2)--(0, 4) node[left]{$y$};
        % 图像
        \draw[line width=1.5pt] (-1.5, 3) -- (3, 3) node[above right]{$(x,y)$} -- (3, -1.5);
        % 填充
        \fill[pattern=north east lines] (-1.5, 3)--(3, 3)--(3, -1.5)--(-1.5, -1.5);
        % 原点标记
        \node at (0, 0) [below left, fill=white] {$O$};
    \end{tikzpicture}

    \caption{}
    \label{fig:F(x,y)}
\end{figure}

分布函数 $F(x,y)$ 具有如下性质:

\setcounter{propertyname}{0}

\begin{property}
    $0 \leqslant F(x,y) \leqslant 1$,且
    \begin{gather*}
        F(-\infty, -\infty) = \lim_{\substack{x \to -\infty \\ y \to -\infty}} F(x,y) = 0 \\
        F(+\infty, +\infty) = \lim_{\substack{x \to +\infty \\ y \to +\infty}} F(x,y) = 1
    \end{gather*}
    对于任意固定的 $x$,有 $F(x,-\infty) = \displaystyle\lim_{y \to -\infty} F(x,y) = 0$.\newline
    对于任意固定的 $y$,有 $F(-\infty,y) = \displaystyle\lim_{x \to -\infty} F(x,y) = 0$.
\end{property}

\begin{property}
    $F(x,y)$ 对于每个变量都是单调不减函数.即对于任意固定的 $y$,当 $x_1 < x_2$ 时,有 $F(x_1, y) \leqslant F(x_2, y)$;对于任意固定的 $x$,当 $y_1 < y_2$ 时,有 $F(x, y_1) \leqslant F(x, y_2)$.
\end{property}

\begin{property}
    $F(x,y)$ 关于 $x$ 右连续,关于 $y$ 右连续,即
    \begin{gather*}
        F(x^+, y) = F(x, y) \\
        F(x, y^+) = F(x, y)
    \end{gather*}
\end{property}

\begin{property}
    对于任意的 $x_1 < x_2, y_1 < y_2$,有
    $$
    P\{x_1 < X \leqslant x_2, y_1 < Y \leqslant y_2\} = F(x_2, y_2) - F(x_1, y_2) - F(x_2, y_1) + F(x_1, y_1)
    $$
\end{property}

\subsection{边缘分布}

\begin{definition}
    设二维随机变量 $(X,Y)$ 的分布函数为 $F(x,y)$,记随机变量 $X$ 的分布函数为 $F_{X}(x)$,随机变量 $Y$ 的分布函数为 $F_{Y}(y)$,分别称为二维随机变量 $(X,Y)$ 关于 $X$ 和关于 $Y$ 的\textbf{边缘分布函数}.
\end{definition}

\begin{gather*}
    F_{X}(x) = P\{X \leqslant x\} = P\{X \leqslant x, Y < +\infty\} = F(x, +\infty), \quad x \in \mathbf{R} \\
    F_{Y}(y) = P\{Y \leqslant y\} = P\{X < +\infty, Y \leqslant y\} = F(+\infty, y), \quad y \in \mathbf{R}
\end{gather*}

\subsection{随机变量的独立性}

\begin{definition}
    设二维随机变量 $(X,Y)$ 的分布函数及其关于 $X$ 和关于 $Y$ 的边缘分布函数分别为 $F(x,y), F_{X}(x), F_{Y}(y)$,如果对于任意实数 $x,y$,都有 $F(x,y)=F_{X}(x)\,F_{Y}(y)$,则称随机变量 $X$ 与 $Y$ 是\textbf{相互独立}的.
\end{definition}

\section{二维离散型随机变量}

\subsection{二维离散型随机变量及其概率分布}

\begin{definition}
    如果二维随机变量 $(X,Y)$ 所有可能取的值是有限对或可列无限对,则称 $(X,Y)$ 是\textbf{二维离散型随机变量}.
\end{definition}

设 $(X,Y)$ 所有可能取的值为 $(x_i,y_j)(i,j=1,2,\cdots)$,记事件 $\{X=x_i\}$ 与 $\{Y=y_j\}$ 的交事件为 $\{X=x_i, Y=y_j\}$.

\begin{definition}
    设 $(X,Y)$ 所有可能取的值为 $(x_i,y_j)(i,j=1,2,\cdots)$,如果
    \begin{equation} \label{equation:P(X,Y)}
        P\{X=x_i, Y=y_j\} = p_{ij}, \; i,j=1,2,\cdots
    \end{equation}
    且有
    \begin{gather*}
        p_{ij} \geqslant 0 \\
        \sum_{i=1}^\infty \sum_{j=1}^\infty p_{ij} = 1
    \end{gather*}
    则称式 \eqref{equation:P(X,Y)} 为二维离散型随机变量 $(X,Y)$ 的\textbf{概率分布},或称为随机变量 $X$ 和随机变量 $Y$ 的\textbf{联合概率分布}或\textbf{联合分布律}.
\end{definition}

$(X,Y)$ 的概率分布可以用如下的表格表示:

\begin{table}[H]
    \centering

    \begin{tabular}{c | c c c c c}
        \hline
        \diagbox{$X$}{$Y$} & $y_1$ & $y_2$ & $\cdots$ & $y_j$ & $\cdots$ \\
        \hline
        $x_1$ & $p_{11}$ & $p_{12}$ & $\cdots$ & $p_{1j}$ & $\cdots$ \\
        $x_2$ & $p_{21}$ & $p_{22}$ & $\cdots$ & $p_{2j}$ & $\cdots$ \\
        $\vdots$ & $\vdots$ & $\vdots$ & & $\vdots$ & \\
        $x_i$ & $p_{i1}$ & $p_{i2}$ & $\cdots$ & $p_{ij}$ & $\cdots$ \\
        $\vdots$ & $\vdots$ & $\vdots$ & & $\vdots$ & \\
        \hline
    \end{tabular}
\end{table}

如果二维随机变量 $(X,Y)$ 的概率分布为 $P\{X=x_i, Y=y_j\} = p_{ij} \, (i,j=1,2,\cdots$),则 $(X,Y)$ 的分布函数为
$$
F(X,Y) = P\{X \leqslant x, Y \leqslant y\} = \sum_{x_i \leqslant x} \sum_{y_j \leqslant y} p_{ij}
$$

\subsection{边缘概率分布}

设二维离散型随机变量 $(X,Y)$ 的概率分布为
$$
P\{X=x_i, Y=y_j\} = p_{ij}, \; i,j=1,2,\cdots
$$
则有
$$
P\{X = x_i\} = P\{X=x_i, \bigcup_{j=1}^\infty \{Y=y_j\}\} = P\{\bigcup_{j=1}^\infty \{X=x_i, Y=y_j\}\}
$$
由于事件 $\{X=x_i, Y=y_j\}(j=1,2,\cdots)$ 是互不相容的,因此
$$
P\{X = x_i\} = \sum_{j=1}^\infty P\{X = x_i, Y = y_j\} = \sum_{j=1}^\infty p_{ij}
$$
记 $\displaystyle\sum_{j=1}^\infty p_{ij} = p_{i\cdot}$,则有
$$
P\{X=x_i\}=p_{i\cdot}, \; i=1,2,\cdots
$$

同理可得
$$
P\{Y = y_j\} = \sum_{i=1}^\infty P\{X = x_i, Y = y_j\} = \sum_{i=1}^\infty p_{ij} = p_{\cdot j}, \; j=1,2,\cdots
$$

\begin{definition}
    设二维离散型随机变量 $(X,Y)$ 的概率分布为
    $$
    P\{X = x_i, Y = y_j\} = p_{ij}, \; i,j=1,2,\cdots
    $$
    随机变量 $X$ 和 $Y$ 的概率分布
    \begin{gather*}
        P\{X = x_i\} = p_{i\cdot}, \; i=1,2,\cdots\\
        P\{Y = y_j\} = p_{\cdot j}, \; j=1,2,\cdots
    \end{gather*}
    分别称为 $(X,Y)$ 关于 $X$ 和关于 $Y$ 的\textbf{边缘概率分布}或\textbf{边缘分布律}.
\end{definition}

\subsection{随机变量的独立性}

设二维离散型随机变量 $(X,Y)$ 的概率分布为
$$
P\{X = x_i, Y = y_j\} = p_{ij}, \; i,j=1,2,\cdots
$$
$(X,Y)$ 关于 $X$ 和关于 $Y$ 的边缘概率分布依次为
\begin{gather*}
    P\{X=x_i\}=p_{i\cdot}, \; i=1,2,\cdots\\
    P\{Y=y_j\}=p_{\cdot j}, \; j=1,2,\cdots
\end{gather*}
则随机变量 $X$ 和 $Y$ 相互独立的充分必要条件是:对任意的 $i,j$,都有
$$
P\{X = x_i, Y = y_j\} = P\{X=x_i\} \, P\{Y=y_j\}, \; i,j=1,2,\cdots
$$
即
$$
p_{ij} = p_{i \cdot} \, p_{\cdot j}, \; i,j=1,2,\cdots
$$

\section{二维连续型随机变量}

\subsection{二维连续型随机变量及其概率密度}

\begin{definition}
    设二维随机变量 $(X,Y)$ 的分布函数为 $F(x,y)$,如果存在非负的二元函数 $f(x,y)$,对于任意实数 $x,y$,有
    $$
    F(x,y) = \int_{-\infty}^x \int_{-\infty}^y f(u,v) \, \text{d}u \text{d}v
    $$
    则称 $(X,Y)$ 为\textbf{二维连续型随机变量},$f(x,y)$ 称为二维连续型随机变量 $(X,Y)$ 的\textbf{概率密度},或称为随机变量 $X$ 和 $Y$ 的\textbf{联合概率密度}.
\end{definition}

概率密度 $f(x,y)$ 具有下列性质:

\setcounter{propertyname}{0}

\begin{property} \label{property: density-1}
    $f(x,y) \geqslant 0$
\end{property}

\begin{property} \label{property: density-2}
    $\displaystyle\int_{-\infty}^{+\infty} \int_{-\infty}^{+\infty} f(x,y) \, \text{d}x \text{d}y = 1$
\end{property}

\begin{property}
    如果 $f(x,y)$ 在点 $(x,y)$ 处连续,则
    $$
    f(x,y) = \dfrac{\partial^2 F(x,y)}{\partial x \partial y}
    $$
\end{property}

\begin{property} \label{property: density-4}
    设 $G$ 是 $xOy$ 平面上的一个区域,则
    $$
    P\{(X,Y) \in G\} = \underset{G}{\iint} f(x,y) \, \text{d}x \text{d}y
    $$
\end{property}

$z=f(x,y)$ 表示空间 $Oxyz$ 中的一张曲面. 性质 \ref*{property: density-1} 和性质 \ref*{property: density-2} 表明,曲面 $z=f(x,y)$ 位于 $xOy$ 平面上方,介于它和 $xOy$ 平面之间的体积为 1. 性质 \ref*{property: density-4} 表示,随机点 $(X,Y)$ 落在区域 $G$ 内的概率 $P\{(X,Y) \in G\}$ 等于以 $G$ 为底、以曲面 $z=f(x,y)$ 为顶的曲顶柱体体积的数值.

\subsection{边缘概率密度}

\begin{definition}
    设二维连续型随机变量 $(X,Y)$ 的概率密度为 $f(x,y), (x,y) \in \mathbf{R}^2$,将一元函数
    \begin{gather*}
        f_{X}(x) = \int_{-\infty}^{+\infty} f(x,y) \, \text{d}y, \ x \in \mathbf{R} \\
        f_{Y}(y) = \int_{-\infty}^{+\infty} f(x,y) \, \text{d}x, \ y \in \mathbf{R}
    \end{gather*}
    分别称为二维随机变量 $(X,Y)$ 关于 $X$ 和关于 $Y$ 的\textbf{边缘概率密度}.
\end{definition}

\subsection{随机变量的独立性}

设二维连续型随机变量 $(X,Y)$ 的概率密度为 $f(x,y), (x,y) \in \mathbf{R}^2$,$(X,Y)$ 关于 $X$ 和关于 $Y$ 的边缘概率密度分别为 $f_{X}(x)$ 和 $f_{Y}(y)$,则随机变量 $X$ 和 $Y$ 相互独立的充分必要条件是:对任意 $x,y \in \mathbf{R}$,都有 $f(x,y)=f_{X}(x) \, f_{Y}(y)$.

\subsection{二维均匀分布}

设 $D$ 是 $xOy$ 平面上的有界区域,其面积为 $A$. 如果二维随机变量 $(X,Y)$ 具有概率密度
$$
f(x,y)=\begin{cases}
    \dfrac{1}{A} & (x,y) \in D \\[0.5em]
    0 & 其他
\end{cases}
$$
则称 $(X,Y)$ 在区域 $D$ 上服从\textbf{均匀分布}.

\subsection{二维正态分布}

设二维随机变量 $(X,Y)$ 的概率密度为
$$
f(x,y) = \dfrac{1}{2 \pi \sigma_1 \sigma_2 \sqrt{1-\rho^2}} e^{\frac{-1}{2(1-\rho^2)} \left[ \frac{(x-\mu_1)^2}{\sigma_1^2} - 2 \rho \frac{(x-\mu_1)(y-\mu_2)}{\sigma_1 \sigma_2} + \frac{(y-\mu_2)^2}{\sigma_2^2} \right]},\ (x,y)\in \mathbf{R}^2
$$
其中 $\mu_1,\mu_2,\sigma_1,\sigma_2,\rho$ 都是常数,且 $\sigma_1 > 0, \sigma_2 > 0, -1 < \rho < 1$,则称 $(X,Y)$ 服从参数为 $\mu_1,\mu_2,\sigma_1,\sigma_2,\rho$ 的\textbf{二维正态分布},记作 $(X,Y) \sim N(\mu_1,\mu_2,\sigma_1^2,\sigma_2^2,\rho)$.

\begin{conclusion}
    设 $(X,Y) \sim N(\mu_1,\mu_2,\sigma_1^2,\sigma_2^2,\rho)$,则 $(X,Y)$ 关于 $X$ 和关于 $Y$ 的边缘概率密度分别为
    \begin{gather*}
        f_{X}(x) = \dfrac{1}{\sqrt{2\pi} \sigma_1} e^{-\frac{(x-\mu_1)^2}{2 \sigma_1^2}}, -\infty < x < +\infty \\
        f_{Y}(y) = \dfrac{1}{\sqrt{2\pi} \sigma_2} e^{-\frac{(y-\mu_2)^2}{2 \sigma_2^2}}, -\infty < y < +\infty
    \end{gather*}
\end{conclusion}

\begin{myproof}
    $$
    \begin{aligned}
        & \frac{-1}{2(1-\rho^2)} \left[ \frac{(x-\mu_1)^2}{\sigma_1^2} - 2 \rho \frac{(x-\mu_1)(y-\mu_2)}{\sigma_1 \sigma_2} + \frac{(y-\mu_2)^2}{\sigma_2^2} \right] \\
        = & \frac{-1}{2(1-\rho^2)} \left\{ \frac{(x-\mu_1)^2}{\sigma_1^2} + \left[ \frac{(y-\mu_2)^2}{\sigma_2^2} - 2 \rho \frac{(x-\mu_1)(y-\mu_2)}{\sigma_1 \sigma_2} + \rho^2 \frac{(x-\mu_1)^2}{\sigma_1^2} \right] - \rho^2 \frac{(x-\mu_1)^2}{\sigma_1^2} \right\} \\
        = & \frac{-1}{2(1-\rho^2)} \left\{ \frac{(1-\rho^2)(x-\mu_1)^2}{\sigma_1^2} + \left( \dfrac{y-\mu_2}{\sigma_2} - \rho \dfrac{x-\mu_1}{\sigma_1} \right)^2 \right\} \\
        = & -\frac{(x-\mu_1)^2}{2\sigma_1^2} - \dfrac{1}{2(1-\rho^2)} \left( \dfrac{y-\mu_2}{\sigma_2} - \rho \dfrac{x-\mu_1}{\sigma_1} \right)^2
    \end{aligned}
    $$
    因此
    $$
    \begin{aligned}
        f_{X}(x) &= \int_{-\infty}^{+\infty} f(x,y) \, \text{d}y \\
        &= \dfrac{1}{2 \pi \sigma_1 \sigma_2 \sqrt{1-\rho^2}} e^{-\frac{(x-\mu_1)^2}{2 \sigma_1^2}} \int_{-\infty}^{+\infty} e^{\frac{-1}{2(1-\rho^2)} \left( \frac{y-\mu_2}{\sigma_2} - \rho \frac{x-\mu_1}{\sigma_1} \right)^2} \, \text{d}y
    \end{aligned}
    $$
    对于任意给定的实数 $x$,令 $t = \dfrac{1}{\sqrt{1-\rho^2}} \left( \dfrac{y-\mu_2}{\sigma_2} - \rho \dfrac{x-\mu_1}{\sigma_1} \right)$,则
    $$
    \text{d}t = \dfrac{1}{\sigma_2 \sqrt{1-\rho^2}} \text{d}y
    $$
    因此
    $$
    \begin{aligned}
        f_{X}(x) &= \dfrac{1}{2 \pi \sigma_1} e^{-\frac{(x-\mu_1)^2}{2 \sigma_1^2}} \int_{-\infty}^{+\infty} e^{-\frac{t^2}{2}} \dfrac{1}{\sigma_2 \sqrt{1-\rho^2}} \text{d}y \\
        &= \dfrac{1}{\sqrt{2\pi} \sigma_1} e^{-\frac{(x-\mu_1)^2}{2 \sigma_1^2}} \int_{-\infty}^{+\infty} \dfrac{1}{\sqrt{2\pi}} e^{-\frac{t^2}{2}} \, \text{d}t
    \end{aligned}
    $$
    因为
    $$
    \int_{-\infty}^{+\infty} \dfrac{1}{\sqrt{2\pi}} e^{-\frac{t^2}{2}} \, \text{d}t = 1
    $$
    所以
    $$
    f_{X}(x) = \dfrac{1}{\sqrt{2\pi} \sigma_1} e^{-\frac{(x-\mu_1)^2}{2 \sigma_1^2}}
    $$

    同理可得
    $$
    f_{Y}(y) = \dfrac{1}{\sqrt{2\pi} \sigma_2} e^{-\frac{(y-\mu_2)^2}{2 \sigma_2^2}}
    $$
\end{myproof}

由此可知,若 $(X,Y) \sim N(\mu_1,\mu_2,\sigma_1^2,\sigma_2^2,\rho)$,则 $(X,Y)$ 关于 $X$ 和关于 $Y$ 的边缘分布都是一维正态分布,且有 $X \sim N(\mu_1,\sigma_1^2)$,$Y \sim N(\mu_2,\sigma_2^2)$. $(X,Y)$ 的分布与参数 $\rho$ 有关,对于不同的 $\rho$,有不同的二维正态分布,但 $(X,Y)$ 关于 $X$ 和关于 $Y$ 的边缘分布都与 $\rho$ 无关.

上述结论还表明,仅仅根据关于 $X$ 和关于 $Y$ 的边缘分布,一般是不能确定随机变量 $X$ 和 $Y$ 的联合分布的.

\begin{conclusion}
    设 $(X,Y) \sim N(\mu_1,\mu_2,\sigma_1^2,\sigma_2^2,\rho)$,则 $X$ 与 $Y$ 相互独立的充分必要条件是 $\rho=0$.
\end{conclusion}

\begin{myproof}
    $$
    f_{X}(x) \, f_{Y}(y) = \dfrac{1}{2 \pi \sigma_1 \sigma_2} e^{-\frac{1}{2} \left[ \frac{(x-\mu_1)^2}{2 \sigma_1^2} + \frac{(y-\mu_2)^2}{2 \sigma_2^2} \right]}
    $$

    先证充分性. 如果 $\rho=0$,则对于任意实数 $x$ 和 $y$,都有 $f(x,y) = f_{X}(x) \, f_{Y}(y)$,因此 $X$ 和 $Y$ 相互独立. 充分性得证.

    再证必要性. 如果 $X$ 和 $Y$ 相互独立,由于 $f(x,y), f_{X}(x), f_{Y}(y)$ 都是连续函数,因此对于任意实数 $x$ 和 $y$,都有 $f(x,y) = f_{X}(x) \, f_{Y}(y)$. 如果取 $x=\mu_1, y=\mu_2$,则有
    $$
    \begin{cases}
        f(\mu_1,\mu_2) = \dfrac{1}{2 \pi \sigma_1 \sigma_2 \sqrt{1-\rho^2}} \\[0.5em]
        f_{X}(\mu_1) \, f_{Y}(\mu_2) = \dfrac{1}{2 \pi \sigma_1 \sigma_2} \\[0.5em]
        f(\mu_1,\mu_2) = f_{X}(\mu_1) \, f_{Y}(\mu_2)
    \end{cases}
    $$
    从而 $\rho=0$. 必要性得证.
\end{myproof}

\section{条件分布}

\subsection{离散型随机变量的条件分布}

\begin{definition}
    设 $(X,Y)$ 是二维离散型随机变量,对于固定的 $j$,如果 $P\{Y = y_j\} = p_{\cdot j} > 0$,则称
    $$
    P\{X=x_i | Y=y_j\} = \dfrac{P\{X=x_i, Y=y_j\}}{P\{Y=y_j\}} = \dfrac{p_{ij}}{p_{\cdot j}},\ i=1,2,\cdots
    $$
    为在条件 $Y=y_j$ 下随机变量 $X$ 的\textbf{条件概率分布}.
    
    对于固定的 $i$,如果 $P\{X=x_i\} = p_{i \cdot} > 0$,则称
    $$
    P\{Y=y_j | X=x_i\} = \dfrac{P\{X=x_i,Y=y_j\}}{P\{X=x_i\}} = \dfrac{p_{ij}}{p_{i \cdot}},\ j=1,2,\cdots
    $$
    为在条件 $X=x_i$ 下随机变量 $Y$ 的\textbf{条件概率分布}.
\end{definition}