\chapter{样本及样本函数的分布}

\section{总体与样本}

\subsection{总体}

在数理统计中,所研究对象的全体称为\textbf{总体},总体中的每个元素称为\textbf{个体}.

总体中所包含的个体总数叫做\textbf{总体容量}.如果一个总体的容量是有限的,则叫做\textbf{有限总体},否则叫做\textbf{无限总体}.

在具体问题中,人们关心的往往不是总体的一切方面,而是它的某一项数量指标 $X$ 以及它在总体中的分布情况.如果 $X$ 在总体中的分布情况可以用一个概率分布来表示,那么 $X$ 就可以看成是服从这一概率分布的随机变量.将表示总体的这项数量指标的随机变量 $X$ 可能取的值的全体作为总体,称作总体 $X$,而 $X$ 可能取的每一个数值 $x$ 都是一个个体.

总体 $X$ 的分布函数 $F(x)$ 叫做总体的分布函数,总体 $X$ 的数字特征叫做总体的数字特征.

如果 $X$ 是离散型随机变量,则其概率分布叫做离散型总体 $X$ 的概率分布;如果 $X$ 是连续型随机变量,则其概率密度叫做连续型总体 $X$ 的概率密度;并将它们和总体 $X$ 的分布函数统称为总体的分布.

\subsection{简单随机样本}

从总体中抽取若干个个体的过程称为\textbf{抽样},抽样结果得到总体 $X$ 的一组试验数据(或观测值)称为\textbf{样本},样本中所含个体的数量称为\textbf{样本容量}.

为了使得样本能很好地反映总体的情况,抽样必须满足以下两个条件:
\begin{enumerate}
    \item 随机性:为了使样本具有充分的代表性,抽样必须是随机的,总体的每一个个体都有同等的机会被抽取到.
    \item 独立性:各次抽取必须是相互独立的,每次抽样的结果既不影响其他各次抽样的结果,也不受其他各次抽样结果的影响.
\end{enumerate}

满足随机性和独立性的抽样方法称为\textbf{简单随机抽样},由此得到的样本称为\textbf{简单随机样本}.

从总体中进行放回抽样是简单随机抽样.从无限总体中抽取一个个体后不会影响总体的分布,因此对无限总体进行不放回抽样是简单随机抽样.