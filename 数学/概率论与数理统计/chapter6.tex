\chapter{样本及样本函数的分布}

\section{总体与样本}

\subsection{总体}

在数理统计中,所研究对象的全体称为\textbf{总体},总体中的每个元素称为\textbf{个体}.

总体中所包含的个体总数叫做\textbf{总体容量}.如果一个总体的容量是有限的,则叫做\textbf{有限总体},否则叫做\textbf{无限总体}.

在具体问题中,人们关心的往往不是总体的一切方面,而是它的某一项数量指标 $X$ 以及它在总体中的分布情况.如果 $X$ 在总体中的分布情况可以用一个概率分布来表示,那么 $X$ 就可以看成是服从这一概率分布的随机变量.将表示总体的这项数量指标的随机变量 $X$ 可能取的值的全体作为总体,称作总体 $X$,而 $X$ 可能取的每一个数值 $x$ 都是一个个体.

总体 $X$ 的分布函数 $F(x)$ 叫做总体的分布函数,总体 $X$ 的数字特征叫做总体的数字特征.

如果 $X$ 是离散型随机变量,则其概率分布叫做离散型总体 $X$ 的概率分布;如果 $X$ 是连续型随机变量,则其概率密度叫做连续型总体 $X$ 的概率密度;并将它们和总体 $X$ 的分布函数统称为总体的分布.

\subsection{简单随机样本}

从总体中抽取若干个个体的过程称为\textbf{抽样},抽样结果得到总体 $X$ 的一组试验数据(或观测值)称为\textbf{样本},样本中所含个体的数量称为\textbf{样本容量}.

为了使得样本能很好地反映总体的情况,抽样必须满足以下两个条件:
\begin{enumerate}
    \item 随机性:为了使样本具有充分的代表性,抽样必须是随机的,总体的每一个个体都有同等的机会被抽取到.
    \item 独立性:各次抽取必须是相互独立的,每次抽样的结果既不影响其他各次抽样的结果,也不受其他各次抽样结果的影响.
\end{enumerate}

满足随机性和独立性的抽样方法称为\textbf{简单随机抽样},由此得到的样本称为\textbf{简单随机样本}.

\begin{itemize}
    \item 从总体中进行放回抽样是简单随机抽样.
    \item 从无限总体中抽取一个个体后不会影响总体的分布,因此从无限总体中进行不放回抽样是简单随机抽样.
    \item 从有限总体中进行不放回抽样,当总体容量 $N$ 很大而样本容量 $n$ 较小时,可以近似看做放回抽样,即可以近似看做简单随机抽样.
\end{itemize}

从总体中抽取容量为 $n$ 的样本,就是对表示总体的随机变量 $X$ 随机地、独立地进行 $n$ 次试验,第 $i$ 次试验的结果可以看做一个随机变量 $X_i \, (i=1,2,\cdots,n)$,$n$ 次试验的结果就是 $n$ 个随机变量 $X_1,X_2,\cdots,X_n$,这些随机变量相互独立,并且与总体 $X$ 服从相同的分布.

\begin{definition}
    设总体 $X$ 是具有某一概率分布的随机变量,如果随机变量 $X_1,X_2,\cdots,X_n$ 相互独立,且都与 $X$ 具有相同的概率分布,则称 $X_1,X_2,\cdots,X_n$ 为来自总体 $X$ 的\textbf{简单随机样本},简称为\textbf{样本},$n$ 称为\textbf{样本容量}.在对总体 $X$ 进行一次具体的抽样并做观测后,得到样本 $X_1,X_2,\cdots,X_n$ 的确切数值 $x_1,x_2,\cdots,x_n$,称为\textbf{样本观察值}(或\textbf{观测值}),简称为\textbf{样本值}.
\end{definition}

如果从总体 $X$ 中抽取到样本观测值 $x_1,x_2,\cdots,x_n$,则可以认为是 $n$ 个相互独立的事件 $\{X_1=x_1\}, \{X_2=x_2\}, \cdots, \{X_n=x_n\}$ 同时发生了.

如果把样本容量为 $n$ 的样本看成是 $n$ 维随机变量 $(X_1,X_2,\cdots,X_n)$,则 $(X_1,X_2,\cdots,X_n)$ 所有可能取值的全体是 $n$ 维空间或它的一个子集,样本观测值是其中的一个点 $(x_1,x_2,\cdots,x_n)$.

如果总体 $X$ 的分布函数为 $F_X(t)$,则样本 $X_1,X_2,\cdots,X_n$ 的联合分布函数为
$$
F(t_1,t_2,\cdots,t_n) = F_X(t_1) \, F_X(t_2) \cdots F_X(t_n) = \prod_{i=1}^{n} F_X(t_i)
$$

如果总体 $X$ 是离散型随机变量,其概率分布为 $P \{ X=t \} = p_X(t)$,则样本 $X_1,X_2,\cdots,X_n$ 的联合概率分布为
$$
P \{ X_1=t_1, X_2=t_2, \cdots, X_n=t_n \} = p_X(t_1) \, p_X(t_2) \cdots p_X(t_n) = \prod_{i=1}^n p_X(t_i)
$$

如果总体 $X$ 是连续型随机变量,其概率密度为 $f_X(t)$,则样本 $X_1,X_2,\cdots,X_n$ 的联合概率密度为
$$
f(t_1,t_2,\cdots,t_n) = f_X(t_1) \, f_X(t_2) \cdots f_X(t_n) = \prod_{i=1}^n f_X(t_i)
$$

\section{直方图}

根据总体 $X$ 的样本观测值 $x_1,x_2,\cdots,x_n$ 作直方图的一般步骤为:

\begin{enumerate}
    \item 找出 $x_1,x_2,\cdots,x_n$ 中的最小值 $x_{(1)}$ 和最大值 $x_{(n)}$,选取略小于 $x_{(1)}$ 的数 $a$ 和略大于 $x_{(n)}$ 的数 $b$;

    \item 根据样本容量确定组数 $k$;

    \item 选取分点
    $$
    a = t_0 < t_1 < \cdots < t_{i-1} < t_i < \cdots < t_k = b
    $$
    把区间 $(a,b)$ 分为 $k$ 个子区间
    $$
    (a, t_1], (t_1, t_2], \cdots, (t_{i-1}, t_i], \cdots, (t_{k-1}, b]
    $$
    第 $i$ 个子区间 $(t_{i-1}, t_i]$ 的长度为
    $$
    \Delta t_i = t_i - t_{i-1}, \, i=1,2,\cdots,k
    $$
    各子区间的长度可以相等,也可以不相等.如果取各子区间的长度相等,则有
    $$
    \Delta t_i = \dfrac{b-a}{k}, \, i=1,2,\cdots,k
    $$
    记 $\Delta t = \dfrac{b-a}{k}$,并把 $\Delta t$ 叫做\textbf{组距}.此时分点为
    $$
    t_i = a + i \Delta t, \, i=1,2,\cdots,k
    $$
    为了方便起见,分点 $t_i$ 应比样本观测值 $x_i$ 多取一位有效数字;

    \item 数出 $x_1,x_2,\cdots,x_n$ 落在每个子区间 $(t_{i-1}, t_i]$ 内的频数 $n_i$,再算出频率
    $$
    f_i = \dfrac{n_i}{n}, \, i=1,2,\cdots,k
    $$

    \item 在 $Ox$ 轴上画出各个分点 $t_i(i=0,1,2,\cdots,k)$,并以各子区间 $(t_{i-1}, t_i]$ 为底,以 $y_i = \dfrac{f_i}{\Delta t_i} (i=0,1,2,\cdots,k)$ 为高作小矩形,这样作出的所有小矩形就构成了直方图.
\end{enumerate}