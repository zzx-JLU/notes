\chapter{随机变量的数字特征}

\section{数学期望}

\subsection{数学期望的概念}

\begin{definition}
    设离散型随机变量 $X$ 的概率分布为 $P\{X=x_k\} = p_k, \; k=1,2,\cdots$,如果无穷级数 $\displaystyle\sum_{k=1}^{\infty} x_k p_k$ 绝对收敛,则称无穷级数 $\displaystyle\sum_{k=1}^{\infty} x_k p_k$ 的和为离散型随机变量 $X$ 的\textbf{数学期望}或\textbf{均值},记作 $E(X)$ 或 $EX$,即
    $$
    E(X) = \sum_{k=1}^{\infty} x_k p_k
    $$

    设连续型随机变量 $X$ 的概率密度为 $f(x)$,如果反常积分 $\displaystyle\int_{-\infty}^{+\infty} x f(x) \, \text{d}x$ 绝对收敛,则称反常积分 $\displaystyle\int_{-\infty}^{+\infty} x f(x) \, \text{d}x$ 的值为连续型随机变量 $X$ 的\textbf{数学期望}或\textbf{均值},记作 $E(X)$ 或 $EX$,即
    $$
    E(X) = \int_{-\infty}^{+\infty} x f(x) \, \text{d}x
    $$
\end{definition}

若随机变量 $X$ 服从参数为 $p$ 的(0-1)分布,即 $X$ 的分布律为

\begin{table}[htbp]
    \centering

    \begin{tabular}{c | c c}
        \hline
        $X$ & 0 & 1 \\
        \hline
        $P$ & $1-p$ & $p$ \\
        \hline
    \end{tabular}
\end{table}
则有
$$
E(X) = 0 \times (1-p) + 1 \times p = p
$$
\\

若随机变量 $X \sim B(n,p)$,即 $X$ 的概率分布为 $P\{X=k\} = C_n^k p^k (1-p)^{n-k}, \; k=0,1,2,\cdots,n$,则
$$
\begin{aligned}
    E(X) &= \sum_{k=0}^n k C_n^k p^k (1-p)^{n-k} \\
    &= \sum_{k=0}^n \dfrac{kn!}{k! (n-k)!} p^k (1-p)^{n-k} \\
    &= \sum_{k=1}^n \dfrac{np(n-1)! \, p^{k-1} (1-p)^{(n-1)-(k-1)}}{(k-1)! \, [(n-1)-(k-1)]!} \\
    &= np[p+(1-p)]^{n-1} \\
    &= np
\end{aligned}
$$

若随机变量 $X \sim P(\lambda)$,即 $X$ 的概率分布为 $P\{X=k\} = \dfrac{\lambda^k e^{-\lambda}}{k!}, \; k=0,1,2,\cdots$,则
$$
\begin{aligned}
    E(X) &= \sum_{k=0}^{\infty} k \dfrac{\lambda^k e^{-\lambda}}{k!} \\
    &= \lambda e^{-\lambda} \sum_{k=1}^{\infty} \dfrac{\lambda^{k-1}}{(k-1)!} \\
    &= \lambda e^{-\lambda} e^{\lambda} \\
    &= \lambda
\end{aligned}
$$

若随机变量 $X$ 在区间 $[a,b]$ 上服从均匀分布,即 $X$ 的概率密度为
$$
f(x) = \begin{cases}
    \dfrac{1}{b-a} & a \leqslant x \leqslant b \\[0.5em]
    0 & \text{其他}
\end{cases}
$$
则
$$
E(X) = \int_{-\infty}^{+\infty} x f(x) \, \text{d}x = \int_a^b \dfrac{x}{b-a} \text{d}x = \dfrac{a+b}{2}
$$

若随机变量 $X$ 服从参数为 $\lambda \, (\lambda>0)$ 的指数分布,即 $X$ 的概率密度为
$$
f(x) = \begin{cases}
    \lambda e^{-\lambda x} & x>0 \\
    0 & x \leqslant 0
\end{cases}
$$
则
$$
\begin{aligned}
    E(X) &= \int_{-\infty}^{+\infty} x f(x) \, \text{d}x \\
    &= \int_0^{+\infty} x \lambda e^{-\lambda x} \, \text{d}x \\
    &= \left. -xe^{-\lambda x} \right|_0^{+\infty} + \int_0^{+\infty} e^{-\lambda x} \, \text{d}x \\
    &= \left. -\dfrac{1}{\lambda} e^{-\lambda x} \right|_0^{+\infty} \\
    &= \dfrac{1}{\lambda}
\end{aligned}
$$

若随机变量 $X \sim N(\mu,\sigma^2)$,即 $X$ 的概率密度为
$$
f(x) = \dfrac{1}{\sqrt{2 \pi} \sigma} e^{-\frac{(x-\mu)^2}{2 \sigma^2}}, \; -\infty < x < +\infty
$$
则
$$
\begin{aligned}
    E(X) &= \int_{-\infty}^{+\infty} x \dfrac{1}{\sqrt{2 \pi} \sigma} e^{-\frac{(x-\mu)^2}{2 \sigma^2}} \text{d}x \\
    & \xlongequal{t = \frac{x-\mu}{\sigma}} \dfrac{1}{\sqrt{2 \pi}} \int_{-\infty}^{+\infty} (\sigma t + \mu) e^{-\frac{t^2}{2}} \text{d}t \\
    &= \dfrac{\sigma}{\sqrt{2 \pi}} \int_{-\infty}^{+\infty} te^{-\frac{t^2}{2}} \text{d}t + \mu \int_{-\infty}^{+\infty} \dfrac{1}{\sqrt{2 \pi}} e^{-\frac{t^2}{2}} \text{d}t \\
    &= \mu
\end{aligned}
$$

\subsection{随机变量函数的数学期望}

\begin{theorem}
    设随机变量 $Y$ 是随机变量 $X$ 的函数:$Y=g(X)$,其中 $g$ 是一元连续函数.

    若 $X$ 是离散型随机变量,其概率分布为 $P\{X=x_k\} = p_k, \; k=1,2,\cdots$,如果无穷级数 $\displaystyle\sum_{k=1}^{\infty} g(x_k) p_k$ 绝对收敛,则随机变量 $Y$ 的数学期望为
    $$
    E(Y) = E[g(X)] = \sum_{k=1}^{\infty} g(x_k) p_k
    $$
\end{theorem}